%% Преамбула TeX-файла

% 1. Стиль и язык
\documentclass[utf8x, 12pt]{G7-32} % Стиль (по умолчанию будет 14pt)

% Остальные стандартные настройки убраны в preamble-std.tex
\sloppy

% 1. Настройки стиля ГОСТ 7-32
% Для начала определяем, хотим мы или нет, чтобы рисунки и таблицы нумеровались в пределах раздела, или нам нужна сквозная нумерация.
% А не забыл ли автор букву 't' ?
\EqInChapter % формулы будут нумероваться в пределах раздела
\TableInChapter % таблицы будут нумероваться в пределах раздела
\PicInChapter % рисунки будут нумероваться в пределах раздела

% 2. Добавляем гипертекстовое оглавление в PDF
\usepackage[
bookmarks=true, colorlinks=true, unicode=true,
urlcolor=black,linkcolor=black, anchorcolor=black,
citecolor=black, menucolor=black, filecolor=black,
]{hyperref}

\usepackage{amsmath}
\usepackage{mathtext}
\usepackage{minted}

% 3. Изменение начертания шрифта --- после чего выглядит таймсоподобно.
% apt-get install scalable-cyrfonts-tex

\IfFileExists{cyrtimes.sty}
    {
        \usepackage{cyrtimespatched}
    }
    {
        % А если Times нету, то будет CM...
    }


% 4. Прочие полезные пакеты.
\usepackage{underscore} % Ура! Теперь можно писать подчёркивание.
                        % И нельзя использовать подчёркивание в файлах.
                        % Выбирай, но осторожно.

\usepackage{graphicx}   % Пакет для включения рисунков

 % 5. Любимые команды
\newcommand{\Code}[1]{\textbf{#1}}

% 6. Поля
% С такими оно полями оно работает по-умолчанию:
% \RequirePackage[left=20mm,right=10mm,top=20mm,bottom=20mm,headsep=0pt]{geometry}
% Если вас тошнит от поля в 10мм --- увеличивайте до 20-ти, ну и про переплёт не забывайте:
\geometry{right=20mm}
\geometry{left=30mm}


% 7. Tikz
\usepackage{tikz}
\usetikzlibrary{arrows,positioning,shadows}

% 8 Листинги

\usepackage{listings}
\usepackage{caption}

% Значения по умолчанию
\lstset{
  basicstyle= \footnotesize,
  breakatwhitespace=true,% разрыв строк только на whitespacce
  breaklines=true,       % переносить длинные строки
%   captionpos=b,          % подписи снизу -- вроде не надо
  inputencoding=koi8-r,
  numbers=left,          % нумерация слева
  numberstyle=\footnotesize,
  showspaces=false,      % показывать пробелы подчеркиваниями -- идиотизм 70-х годов
  showstringspaces=false,
  showtabs=false,        % и табы тоже
  stepnumber=1,
  tabsize=4,              % кому нужны табы по 8 символов?
  frame=single
}

% Стиль для псевдокода: строчки обычно короткие, поэтому размер шрифта побольше
\lstdefinestyle{pseudocode}{
  basicstyle=\small,
  keywordstyle=\color{black}\bfseries\underbar,
  language=Pseudocode,
  numberstyle=\footnotesize,
  commentstyle=\footnotesize\it
}

% Стиль для обычного кода: маленький шрифт
\lstdefinestyle{realcode}{
  basicstyle=\scriptsize,
  numberstyle=\footnotesize
}

% Стиль для коротких кусков обычного кода: средний шрифт
\lstdefinestyle{simplecode}{
  basicstyle=\footnotesize,
  numberstyle=\footnotesize
}

% Стиль для BNF
\lstdefinestyle{grammar}{
  basicstyle=\footnotesize,
  numberstyle=\footnotesize,
  stringstyle=\bfseries\ttfamily,
  language=BNF
}

% Определим свой язык для написания псевдокодов на основе Python
\lstdefinelanguage[]{Pseudocode}[]{Python}{
  morekeywords={each,empty,wait,do},% ключевые слова добавлять сюда
  morecomment=[s]{\{}{\}},% комменты {а-ля Pascal} смотрятся нагляднее
  literate=% а сюда добавлять операторы, которые хотите отображать как мат. символы
    {->}{\ensuremath{$\rightarrow$}~}2%
    {<-}{\ensuremath{$\leftarrow$}~}2%
    {:=}{\ensuremath{$\leftarrow$}~}2%
    {<--}{\ensuremath{$\Longleftarrow$}~}2%
}[keywords,comments]

% Свой язык для задания грамматик в BNF
\lstdefinelanguage[]{BNF}[]{}{
  morekeywords={},
  morecomment=[s]{@}{@},
  morestring=[b]",%
  literate=%
    {->}{\ensuremath{$\rightarrow$}~}2%
    {*}{\ensuremath{$^*$}~}2%
    {+}{\ensuremath{$^+$}~}2%
    {|}{\ensuremath{$|$}~}2%
}[keywords,comments,strings]

\usepackage{color}
\definecolor{lightgray}{rgb}{.9,.9,.9}
\definecolor{darkgray}{rgb}{.4,.4,.4}
\definecolor{purple}{rgb}{0.65, 0.12, 0.82}

\lstdefinelanguage{JavaScript}{
  keywords={typeof, new, true, false, catch, function, return, null, catch, switch, var, if, in, while, do, else, case, break},
  keywordstyle=\color{blue}\bfseries,
  ndkeywords={class, export, boolean, throw, implements, import, this},
  ndkeywordstyle=\color{darkgray}\bfseries,
  identifierstyle=\color{black},
  sensitive=false,
  comment=[l]{//},
  morecomment=[s]{/*}{*/},
  commentstyle=\color{purple}\ttfamily,
  stringstyle=\color{red}\ttfamily,
  morestring=[b]',
  morestring=[b]"
}
% Подписи к листингам на русском языке.
\renewcommand*\thelstnumber{\oldstylenums{\the\value{lstnumber}}}
\renewcommand\lstlistingname{\cyr\CYRL\cyri\cyrs\cyrt\cyri\cyrn\cyrg}
\renewcommand\lstlistlistingname{\cyr\CYRL\cyri\cyrs\cyrt\cyri\cyrn\cyrg\cyri}

% Произвольная нумерация списков.
\usepackage{enumerate}

\lstset{
literate={а}{{\selectfont\char224}}1
{б}{{\selectfont\char225}}1
{в}{{\selectfont\char226}}1
{г}{{\selectfont\char227}}1
{д}{{\selectfont\char228}}1
{е}{{\selectfont\char229}}1
{ё}{{\"e}}1
{ж}{{\selectfont\char230}}1
{з}{{\selectfont\char231}}1
{и}{{\selectfont\char232}}1
{й}{{\selectfont\char233}}1
{к}{{\selectfont\char234}}1
{л}{{\selectfont\char235}}1
{м}{{\selectfont\char236}}1
{н}{{\selectfont\char237}}1
{о}{{\selectfont\char238}}1
{п}{{\selectfont\char239}}1
{р}{{\selectfont\char240}}1
{с}{{\selectfont\char241}}1
{т}{{\selectfont\char242}}1
{у}{{\selectfont\char243}}1
{ф}{{\selectfont\char244}}1
{х}{{\selectfont\char245}}1
{ц}{{\selectfont\char246}}1
{ч}{{\selectfont\char247}}1
{ш}{{\selectfont\char248}}1
{щ}{{\selectfont\char249}}1
{ъ}{{\selectfont\char250}}1
{ы}{{\selectfont\char251}}1
{ь}{{\selectfont\char252}}1
{э}{{\selectfont\char253}}1
{ю}{{\selectfont\char254}}1
{я}{{\selectfont\char255}}1
{А}{{\selectfont\char192}}1
{Б}{{\selectfont\char193}}1
{В}{{\selectfont\char194}}1
{Г}{{\selectfont\char195}}1
{Д}{{\selectfont\char196}}1
{Е}{{\selectfont\char197}}1
{Ё}{{\"E}}1
{Ж}{{\selectfont\char198}}1
{З}{{\selectfont\char199}}1
{И}{{\selectfont\char200}}1
{Й}{{\selectfont\char201}}1
{К}{{\selectfont\char202}}1
{Л}{{\selectfont\char203}}1
{М}{{\selectfont\char204}}1
{Н}{{\selectfont\char205}}1
{О}{{\selectfont\char206}}1
{П}{{\selectfont\char207}}1
{Р}{{\selectfont\char208}}1
{С}{{\selectfont\char209}}1
{Т}{{\selectfont\char210}}1
{У}{{\selectfont\char211}}1
{Ф}{{\selectfont\char212}}1
{Х}{{\selectfont\char213}}1
{Ц}{{\selectfont\char214}}1
{Ч}{{\selectfont\char215}}1
{Ш}{{\selectfont\char216}}1
{Щ}{{\selectfont\char217}}1
{Ъ}{{\selectfont\char218}}1
{Ы}{{\selectfont\char219}}1
{Ь}{{\selectfont\char220}}1
{Э}{{\selectfont\char221}}1
{Ю}{{\selectfont\char222}}1
{Я}{{\selectfont\char223}}1
}


\begin{document}

\frontmatter % выключает нумерацию ВСЕГО; здесь начинаются ненумерованные главы: реферат, введение, глоссарий, сокращения и прочее

% Команды \breakingbeforechapters и \nonbreakingbeforechapters
% управляют разрывом страницы перед главами.
% По-умолчанию страница разрывается.

% \nobreakingbeforechapters
% \breakingbeforechapters

% Также можно использовать \Referat, как в оригинале
\begin{abstract}
Это пример каркаса расчётно-пояснительной записки, желательный к использованию в РПЗ проекта по курсу РСОИ.

Дополняет краткое пособие по графике в Latex.  Данный опус, как и более новые версии этого документа, можно взять по адресу (\url{http://sevik.ru/latex}). Минимально необходимые пакеты Latex, которые должны стоять: mathtext, amssymb, amsmath, icomma, longtable, graphicx, underscore, cmap, hyperref.

Текст в документе носит совершенно абстрактный характер.
\end{abstract}

%%% Local Variables: 
%%% mode: latex
%%% TeX-master: "rpz"
%%% End: 


\tableofcontents

\Defines % Необходимые определения. Вряд ли понадобться
\begin{description}
\item[Распределённый] Слово, которое нельзя употреблять. Но надо протестировать длинные строки в глоссарии.
\end{description}

%%% Local Variables:
%%% mode: latex
%%% TeX-master: "rpz"
%%% End:

\Abbreviations %% Список обозначений и сокращений в тексте
\begin{description}
\item[API] Интерфейс программирования приложений (англ. application programming interface)
\end{description}

%%% Local Variables:
%%% mode: latex
%%% TeX-master: "rpz"
%%% End:


\Introduction

В настоящее время всемирная паутина предоставляет огромное количество информации для пользователя на любой вкус. Миллиарды байт буквально лежать на кончиках пальцев миллионов пользователей. И в наше время не обязательно быть привязанным к компьютеру, чтобы получить доступ к этой информации. Достаточно иметь телефон, планшет или даже фотоаппарат с поддержкой беспроводной технологии доступа в интернет и ты получаешь возможность получить любую информацию за пару прикосновений прохладного стекла твоего девайса. Но протокол HTTP, благодаря которому мы имеем сегодня поистине безграничные возможности, разработан, как говорит название, для передачи гипертекста, то есть, данные приходят клиенту в виде размеченного документа. Это очень удобно при чтении документа с экрана компьютера или другого устройства отображения информации человеку, но совершенно не приемлемо для использования в других автоматизированных системах, когда клинтом является другое приложения. В таком случае язык HTML, который в основном и используется для распространения информации во всемирной паутине, является декоратором необходимых данных. 

При разработке приложений разработчики сталкиваются в первую очередь с проблемой получения и хранения информации. Первая проблема решается путем использования специализированных интерфейсов для получения данных и заботливые разработчики начинают все больше внимания уделять разработке API для того, чтобы их сервисы были удобнее как для простых пользователей, так и для программистов, желающих использовать накопившуюся у них ценную информацию в собственных целях.

Но так бывает далеко не во всех случаях, и в такой ситуации разработчику приходится разгребать кучи HTML кода для фильтрации необходимых данных и отсеивания лишней разметки для того, чтобы превратить тонны неструктурированной информации с просторов паутины в четко структурированные единицы данных. В таком случае очень удобна автоматизация процесса накопления, фильтрации и дальнейшего сохранения необходимой информации на стороне сервиса. 

Целью дипломной работы является создание программного комплекса для предоставления справочной информации для разработчиков о спортивных мероприятиях. Для достижения данной цели необходимо выполнить следующее: 

\begin{itemize}
\item разработать подсистему сбора информации, распространяемой в сети Internet в неструктурированном виде, в формате HTML страниц, работающую в реальном времени;
\item спроектировать базу данных для хранения собранной информации;
\item спроектировать REST API для предоставления доступа к собранной информации в удобном для использования виде;
\item проверить работоспособность системы путем создания приложения, используемого в качестве клиента API.
\end{itemize}

Подсистема сбора информации должна работать в качестве "демона" на сервере и представлять из себя очередь заданий для выполнения задач сбора. Собранная информация должна сохраняться в спроектированной базе данных через API для большей надежности и безопасности системы. Так же API должно предоставлять доступ на чтение для сторонних разработчиков, заинтересованных в хранимой информации. Клиентское приложение должно представлять собой web-сайт, предоставляющий пользователю доступ к статистической и исторической информации о спортивных мероприятиях. 

В данной работе упор делается на футбольную статистику, но разрабатываемая система должна быть легко расширяем для сбора и сохранения информации по другим видам спорта. Таким образом, каждая подсистема должна быть гибка и открыта для расширения.


\mainmatter % это включает нумерацию глав и секций в документе ниже

\chapter{Аналитический раздел}
\label{cha:analysis}
%
% % В начале раздела  можно напомнить его цель
%
В данном разделе производится анализ процессов сбора информации, сохранения её на стороне сервера и распространение её средствами API.
Производится анализ подсистем, входящих в реализуемы программный комплекс, формируются требования к создаваемой системе, выделяются функции её подсистем и описывается взаимодействие между ними.

\section{Общее описание системы}
В данной работе разрабатывается система полного цикла сбора, хранения и распространения информации, для функционирования которой необходимо спроектировать все подсистемы программного комплекса и способы их взаимодействия.

Для достижения поставленной цели необходимо спроектировать и разработать следующие компоненты системы:

\begin{enumerate}
\item подсистему сбора информации, находящихся на просторах интернета в виде страниц HTML, работающую в фоновом режиме;
\item базу данных для хранения собранной информации и быстрого доступа к ней;
\item простой и удобный в использовании REST API для предоставления доступа к собранной информации сторонним приложениям;
\item клиентское приложение, используемого в качестве клиента API для демонстрации возможностей разработанной.
\end{enumerate}

Так же для удобства и простоты использования необходимо разработать доступную документацию разработанного API, для того, чтобы любой желающий разработчик имел возможность легко и быстро воспользоваться услугами сервиса. 

Поскольку конечным пользователем разрабатываемой системы будет некое стороннее приложение, рассмотрим работу клиента системы с точки зрения разработчика-пользователя API. Работа с системой выглядит следующим образом:

\begin{enumerate}
\item разработчик создает приложение, имеющее возможность доступа в интернет;
\item он регистрирует свое приложение в системе для получения доступа к API;
\item используя известный протокол прикладного уровня передачи данных в среде web, приложение обращается по определенному адресу к API;
\item при обмене сообщениями протокола запрашиваемая система(сервер) отсылает ответ клиенту;
\item клиентское приложение использует полученную информацию по своему усмотрению.
\end{enumerate}

\section{Обзор существующих решений}
В настоящее время существует некоторое количество подобных решений, предоставляющих API(REST, SOAP, XML-RPC, и т.д.) для доступа к спортивной информации сторонним приложениям.
Но они ограничены в том или ином виде. Наиболее полные и интересные решения являются платными или условно бесплатными, предоставляющими пробный, ограниченный функционал бесплатно. Бесплатные API решения предоставляют неполную информацию или плохо документированный сервис. В большинстве API отсутствует информации о российских турнирах, что является одним из решающих факторов в пользу создания собственного API.

Наиболее интересными решениями в сфере предоставления спортивной информации являются следующие сервисы:

\begin{itemize}
\item \url{http://developer.espn.com/} - API был разработан для того, чтобы быть простым в использовании. Поддерживает XML, JSON/JSONP форматы ответа сервера. Имеет хорошую документацию. К сожалению нацелен на американского разработчика, так как в основном предоставляет доступ к информации, связанной с наиболее распространенными видам спорта в америке. Имеет условно бесплатный доступ;
\item \url{http://www.footytube.com/openfooty/} - Интересный сервис предоставляющие REST API возвращающий ответ в формате XML. Предоставляет информацию только о футболе. Хорошо документирован. Имеет ограничение в 5000 запросов в день;
\item \url{http://www.championat.com/} - Популярный российский спортивный портал, содержащий большое количество статистической и исторической информации, новостей из мира спорта, информации о текущих и будущих событиях. Имеет широкий круг рассматриваемых видов спорта, охватывающий широкий круг пользователей. Имеет элементы социальной сети. К сожалению не имеет собственного API;
\item ...
\end{itemize}

Ни один из этих сервисов не предоставляет информации о спортивных событиях в режиме "live".

Таким образом, обзор существующих решений показал, что все сервисы, предоставляющие подобный функционал являются довольно узкоспециализированными, ограниченными и не универсальными, а сервисы, предоставляющие необходимые данные в полном или практически полном объеме не предоставляют прямого доступа к ним посредствам API. 

\section{Подсистема сбора данных}
\subsection{Общие представления о системе}
 % Обратите внимание, что включается не ../dia/..., а inc/dia/...
% В Makefile есть соответствующее правило для inc/dia/*.pdf, которое
% берет исходные файлы из ../dia в этом случае.

Подсистема сбора данных должна решать проблемы "переваривания" "сырых" данных из сети и формирования из них базы данных. Таким образом, собранная на просторах интернета информация должна принять удобный для хранения, использования и распространения, структурированный вид. Данных процесс в общем случае представлен на рисунке 1.1.
\begin{figure}
  \centering
  \includegraphics[width=\textwidth]{inc/dia/analysis1-1}
  \caption{Рисунок}
  \label{fig:fig01}
\end{figure}


\subsection{Существующие сервисы сбора информации}

\begin{itemize}
\item \url{http://www.mozenda.com/} - сервис сбора информации в интернете. Является SaaS-приложением для решения схожих проблем поиска информации в сети. Является довольно дорогим сервисом. Предоставляет on-line конструктор web-пауков для сбора информации, что является несомненным плюсом сервиса
\item \url{http://www.fetch.com/} - название сайта говорит само за себя. Является похожим сервисом сбора информации с просторов паутины, но не предоставляет полной информации о своем сервисе и условиях использования на сайте. Для полного ознакомления предлагает связаться со службой поддержки.
\end{itemize}

Приведенный список показывает, что сбор информации в интернете является популярной и сложной задачей. Для универсального решения данной задачи требуется спроектировать сложный программно-аппаратный комплекс, требующий огромных вычислительных мощностей, что выходит за рамки данной работы и может воплотиться в виде развития рассматриваемой темы. 

\subsection{Процесс сбора информации}

Подсистема сбора информации должна работать в режиме 24/7 на некотором сервере и должна предоставлять возможность сбора информации сразу с нескольких предполагаемых источников.
Система сбора должна быть полностью автоматизирована и получать от администратора только правила фильтрации данных. Так как данная система работает в сети интернет, она должна быть готова к "отказам" серверных участников обмена информацией(источников данных), то есть при невозможности выполнения задачи получения страницы, система должна отложить данную задачу либо немедленно повторить процесс. Поскольку системе задаются некоторые правила сбора, она должна фиксировать промежуточные результаты работы в некотором хранилище. Такой подход позволит провести декомпозицию процесса сбора на элементарные единицы работы, что позволит более гибко реагировать на отказы системы, нештатные ситуации, связанные с работой сети и другие возможные сложности при выполнения сбора единицы полезной информации.

Данные требования хорошо решаются путем создания очереди заданий, представляющих собой систему массового обслуживания. Данная система состоит из очереди задач и обслуживающего аппарата. Поскольку процесс сбора единицы информации проходит декомпозицию на подзадачи, данные подзадачи должны ставиться на обслуживание в очередь отдельно. Таким образом мы получаем очередь с разными типами заявок. В случае простого накопления данных очередность выполнения заданий не имеет значения, так как итоговой целью работы очереди является выполнение всех задач. В нашем случае некоторые задачи могут иметь более высокий приоритет, так как должны быть доступны сразу после момента поступления(с некоторым допущением).
Данное требование естественным образом вводит свойство приоритета задачи. Таки образом, мы получаем очередь с приоритетами и, поскольку, данная работа не требует немедленного выполнения(из-за того же допущения, связанного со спецификой распространения информации в сети), то допускается ожидание завершения выполнения заявки, находящейся на обслуживании.
Данные требования приводят нас к решению использования очереди заданий с относительными приоритетами. 

На более низком уровне процесс сбора единицы информации выглядит следующим образом:
\begin{enumerate}
\item Система переходит по некоторой базовой ссылке, ведущей на определенный ресурс в интернете;
\item Полученный ответ система пытается разобрать по некоторым, заранее заданным правилам;
\item В случае нахождения очередной ссылки, в очередь ставится новая задача;
\item Процесс продолжается до тех пор, пока выполнение разбора не приведет к итоговой цели единицы информации;
\item В случае сбоя в процессе выполнения задачи, она ставится в очередь для повторного выполнения;
\end{enumerate}

Данный процесс представлен на рисунке 1.2.
Подсистема сбора данных должна решать проблемы "переваривания" "сырых" данных из сети и формирования из них базы данных. Таким образом, собранная на просторах интернета информация должна принять удобный для хранения, использования и распространения, структурированный вид. Данных процесс в общем случае представлен на рисунке 1.1.
\begin{figure}
  \centering
  \includegraphics[width=\textwidth]{inc/dia/analysis1-2}
  \caption{Рисунок}
  \label{fig:fig02}
\end{figure}

\chapter{Конструкторский раздел}
\label{cha:design}

В данном разделе проектируется новая всячина.

\section{Архитектура всячины}

\paragraph{Проверка} параграфа. Вроде работает.
\paragraph{Вторая проверка} параграфа. Опять работает.

Вот.

\begin{itemize}
\item Это список с <<палочками>>.
\item Хотя он и не по ГОСТ, кажется.
\end{itemize}

\begin{enumerate}
\item Поэтому для списка, начинающегося с заглавной буквы, лучше список с цифрами.
\end{enumerate}

Формула \ref{F:F1} совершено бессмысленна.

%Кстати, при каких-то условиях <<удавалось>> получить двойный скобки вокруг номеров формул. Вопрос исследуется.

\begin{equation}
a= cb
\label{F:F1}
\end{equation}


Окружение \texttt{cases} опять работает (см. \ref{F:F2}), спасибо И. Короткову за исправления..


\begin{equation}
a= \begin{cases}
 3x + 5y + z, \mbox{если хорошо} \\
 7x - 2y + 4z, \mbox{если плохо}\\
 -6x + 3y + 2z, \mbox{если совсем плохо}
\end{cases}
\label{F:F2}
\end{equation}

\section{Подсистема всякой ерунды}

Культурная вставка dot-файлов через утилиту dot2tex (рис.~\ref{fig:fig02}).

% \begin{figure}
%   \centering
% % [width=0.5\textwidth] --- регулировка ширины картинки
%   \includegraphics{inc/dot/cow2}
%   \caption{Рисунок}
%   \label{fig:fig02}
% \end{figure}


\subsection{Блок-схема всякой ерунды}

\subsubsection*{Кстати о заголовках}

У нас есть и \Code{subsubsection}. Только лучше её не нумеровать.

%%% Local Variables:
%%% mode: latex
%%% TeX-master: "rpz"
%%% End:

\chapter{Технологический раздел}
\label{cha:impl}

В данном разделе описано изготовление и требование всячины. Благодаря пакет \Code{underscore} эскейпить подчёркивание  не нужно (\Code{some_function}).

Для вставки кода есть пакет \Code{listings}. К сожалению, пакет \Code{listings} всё ещё
работает криво при появлении в листинге русских букв и кодировке исходников utf-8.
В данном примере он (увы) на лету конвертируется в koi-8 в ходе сборки pdf.

Есть альтернатива \Code{listingsutf8}, однако она работает лишь с
\texttt{\textbackslash lstinputlisting}, но не с окружением \Code{lstlisting}

Вот так можно вставлять псевдокод (питоноподобный язык определен в шаблоне):

\begin{lstlisting}[style=pseudocode,caption={Алгоритм оценки дипломных работ}]
def EvaluateDiplomas():
    for each student in Masters:
        student.Mark := 5
    for each student in Engineers:
        if Good(student):
            student.Mark := 5
        else:
            student.Mark := 4
\end{lstlisting}

Еще в шаблоне определен псевдоязык для BNF:

\begin{lstlisting}[style=grammar,basicstyle=\small,caption={Грамматика}]
  ifstmt -> "if" "(" expression ")" stmt |
            "if" "(" expression ")" stmt1 "else" stmt2
  number -> digit digit*
\end{lstlisting}

В листинге~\ref{lst:sample01} работают русские буквы. Сильная магия. Однако, работает
только во включаемых файлах, прямо в \TeX{} нельзя.

% Обратите внимание, что включается не ../src/..., а inc/src/...
% В Makefile есть соответствующее правило для inc/src/*,
% которое копирует исходные файлы из ../src и конвертирует из UTF-8 в KOI8-R.
% Кстати, поэтому использовать можно только русские буквы и ASCII,
% весь остальной UTF-8 вроде CJK и египетских иероглифов -- нельзя.

% \lstinputlisting[language=C,caption=Пример (\Code{test.c}),label=lst:sample01]{inc/src/test.c}

% Для вставки реального кода лучше использовать \texttt{\textbackslash lstinputlisting} (который понимает
% UTF8) и стили \Code{realcode} либо \Code{simplecode} (в зависимости от размера куска).




Можно также использовать окружение \Code{verbatim}, если \Code{listings} чем-то не
устраивает. Только следует помнить, что табы в нём <<съедаются>>. Существует так же команда \Code{verbatiminput} для вставки файла.

\begin{verbatim}
a_b = a + b; // русский комментарий
if (a_b > 0)
    a_b = 0;
\end{verbatim}

%%% Local Variables:
%%% mode: latex
%%% TeX-master: "rpz"
%%% End:

\chapter{Экспериментальный раздел}
\label{cha:research}

В данном разделе проводятся вычислительные эксперименты.
А на рис.~\ref{fig:spire01} показана схема мыслительного процесса автора...

% \begin{figure}
%   \centering
%   \includegraphics[width=\textwidth]{inc/svg/pic01}
%   \caption{Как страшно жить}
%   \label{fig:spire01}
% \end{figure}


%%% Local Variables:
%%% mode: latex
%%% TeX-master: "rpz"
%%% End:


\backmatter %% Здесь заканчивается нумерованная часть документа и начинаются ссылки и
            %% заключение

\Conclusion % заключение к отчёту

В результате проделанной работы стало ясно, что ничего не ясно...

%%% Local Variables: 
%%% mode: latex
%%% TeX-master: "rpz"
%%% End: 


% % Список литературы при помощи BibTeX
% Юзать так:
%
% pdflatex rpz
% bibtex rpz
% pdflatex rpz

\bibliographystyle{gost780u}
\bibliography{rpz}

%%% Local Variables: 
%%% mode: latex
%%% TeX-master: "rpz"
%%% End: 


\appendix   % Тут идут приложения

\chapter{Картинки}
\label{cha:appendix1}

\begin{figure}
\centering
\caption{Картинка в приложении. Страшная и ужасная.}
\end{figure}

%%% Local Variables: 
%%% mode: latex
%%% TeX-master: "rpz"
%%% End: 

\chapter{Еще картинки}
\label{cha:appendix2}

\begin{figure}
\centering
\caption{Еще одна картинка, ничем не лучше предыдущей. Но надо же как-то заполнить место.}
\end{figure}

%%% Local Variables: 
%%% mode: latex
%%% TeX-master: "rpz"
%%% End: 


\end{document}

%%% Local Variables:
%%% mode: latex
%%% TeX-master: t
%%% End:
