%% Преамбула TeX-файла

% 1. Стиль и язык
\documentclass[utf8x, 12pt]{G7-32} % Стиль (по умолчанию будет 14pt)

% Остальные стандартные настройки убраны в preamble-std.tex
\sloppy

% 1. Настройки стиля ГОСТ 7-32
% Для начала определяем, хотим мы или нет, чтобы рисунки и таблицы нумеровались в пределах раздела, или нам нужна сквозная нумерация.
% А не забыл ли автор букву 't' ?
\EqInChapter % формулы будут нумероваться в пределах раздела
\TableInChapter % таблицы будут нумероваться в пределах раздела
\PicInChapter % рисунки будут нумероваться в пределах раздела

% 2. Добавляем гипертекстовое оглавление в PDF
\usepackage[
bookmarks=true, colorlinks=true, unicode=true,
urlcolor=black,linkcolor=black, anchorcolor=black,
citecolor=black, menucolor=black, filecolor=black,
]{hyperref}

\usepackage{amsmath}
\usepackage{mathtext}
\usepackage{minted}

% 3. Изменение начертания шрифта --- после чего выглядит таймсоподобно.
% apt-get install scalable-cyrfonts-tex

\IfFileExists{cyrtimes.sty}
    {
        \usepackage{cyrtimespatched}
    }
    {
        % А если Times нету, то будет CM...
    }


% 4. Прочие полезные пакеты.
\usepackage{underscore} % Ура! Теперь можно писать подчёркивание.
                        % И нельзя использовать подчёркивание в файлах.
                        % Выбирай, но осторожно.

\usepackage{graphicx}   % Пакет для включения рисунков

 % 5. Любимые команды
\newcommand{\Code}[1]{\textbf{#1}}

% 6. Поля
% С такими оно полями оно работает по-умолчанию:
% \RequirePackage[left=20mm,right=10mm,top=20mm,bottom=20mm,headsep=0pt]{geometry}
% Если вас тошнит от поля в 10мм --- увеличивайте до 20-ти, ну и про переплёт не забывайте:
\geometry{right=20mm}
\geometry{left=30mm}


% 7. Tikz
\usepackage{tikz}
\usetikzlibrary{arrows,positioning,shadows}

% 8 Листинги

\usepackage{listings}
\usepackage{caption}

% Значения по умолчанию
\lstset{
  basicstyle= \footnotesize,
  breakatwhitespace=true,% разрыв строк только на whitespacce
  breaklines=true,       % переносить длинные строки
%   captionpos=b,          % подписи снизу -- вроде не надо
  inputencoding=koi8-r,
  numbers=left,          % нумерация слева
  numberstyle=\footnotesize,
  showspaces=false,      % показывать пробелы подчеркиваниями -- идиотизм 70-х годов
  showstringspaces=false,
  showtabs=false,        % и табы тоже
  stepnumber=1,
  tabsize=4,              % кому нужны табы по 8 символов?
  frame=single
}

% Стиль для псевдокода: строчки обычно короткие, поэтому размер шрифта побольше
\lstdefinestyle{pseudocode}{
  basicstyle=\small,
  keywordstyle=\color{black}\bfseries\underbar,
  language=Pseudocode,
  numberstyle=\footnotesize,
  commentstyle=\footnotesize\it
}

% Стиль для обычного кода: маленький шрифт
\lstdefinestyle{realcode}{
  basicstyle=\scriptsize,
  numberstyle=\footnotesize
}

% Стиль для коротких кусков обычного кода: средний шрифт
\lstdefinestyle{simplecode}{
  basicstyle=\footnotesize,
  numberstyle=\footnotesize
}

% Стиль для BNF
\lstdefinestyle{grammar}{
  basicstyle=\footnotesize,
  numberstyle=\footnotesize,
  stringstyle=\bfseries\ttfamily,
  language=BNF
}

% Определим свой язык для написания псевдокодов на основе Python
\lstdefinelanguage[]{Pseudocode}[]{Python}{
  morekeywords={each,empty,wait,do},% ключевые слова добавлять сюда
  morecomment=[s]{\{}{\}},% комменты {а-ля Pascal} смотрятся нагляднее
  literate=% а сюда добавлять операторы, которые хотите отображать как мат. символы
    {->}{\ensuremath{$\rightarrow$}~}2%
    {<-}{\ensuremath{$\leftarrow$}~}2%
    {:=}{\ensuremath{$\leftarrow$}~}2%
    {<--}{\ensuremath{$\Longleftarrow$}~}2%
}[keywords,comments]

% Свой язык для задания грамматик в BNF
\lstdefinelanguage[]{BNF}[]{}{
  morekeywords={},
  morecomment=[s]{@}{@},
  morestring=[b]",%
  literate=%
    {->}{\ensuremath{$\rightarrow$}~}2%
    {*}{\ensuremath{$^*$}~}2%
    {+}{\ensuremath{$^+$}~}2%
    {|}{\ensuremath{$|$}~}2%
}[keywords,comments,strings]

\usepackage{color}
\definecolor{lightgray}{rgb}{.9,.9,.9}
\definecolor{darkgray}{rgb}{.4,.4,.4}
\definecolor{purple}{rgb}{0.65, 0.12, 0.82}

\lstdefinelanguage{JavaScript}{
  keywords={typeof, new, true, false, catch, function, return, null, catch, switch, var, if, in, while, do, else, case, break},
  keywordstyle=\color{blue}\bfseries,
  ndkeywords={class, export, boolean, throw, implements, import, this},
  ndkeywordstyle=\color{darkgray}\bfseries,
  identifierstyle=\color{black},
  sensitive=false,
  comment=[l]{//},
  morecomment=[s]{/*}{*/},
  commentstyle=\color{purple}\ttfamily,
  stringstyle=\color{red}\ttfamily,
  morestring=[b]',
  morestring=[b]"
}
% Подписи к листингам на русском языке.
\renewcommand*\thelstnumber{\oldstylenums{\the\value{lstnumber}}}
\renewcommand\lstlistingname{\cyr\CYRL\cyri\cyrs\cyrt\cyri\cyrn\cyrg}
\renewcommand\lstlistlistingname{\cyr\CYRL\cyri\cyrs\cyrt\cyri\cyrn\cyrg\cyri}

% Произвольная нумерация списков.
\usepackage{enumerate}

\lstset{
literate={а}{{\selectfont\char224}}1
{б}{{\selectfont\char225}}1
{в}{{\selectfont\char226}}1
{г}{{\selectfont\char227}}1
{д}{{\selectfont\char228}}1
{е}{{\selectfont\char229}}1
{ё}{{\"e}}1
{ж}{{\selectfont\char230}}1
{з}{{\selectfont\char231}}1
{и}{{\selectfont\char232}}1
{й}{{\selectfont\char233}}1
{к}{{\selectfont\char234}}1
{л}{{\selectfont\char235}}1
{м}{{\selectfont\char236}}1
{н}{{\selectfont\char237}}1
{о}{{\selectfont\char238}}1
{п}{{\selectfont\char239}}1
{р}{{\selectfont\char240}}1
{с}{{\selectfont\char241}}1
{т}{{\selectfont\char242}}1
{у}{{\selectfont\char243}}1
{ф}{{\selectfont\char244}}1
{х}{{\selectfont\char245}}1
{ц}{{\selectfont\char246}}1
{ч}{{\selectfont\char247}}1
{ш}{{\selectfont\char248}}1
{щ}{{\selectfont\char249}}1
{ъ}{{\selectfont\char250}}1
{ы}{{\selectfont\char251}}1
{ь}{{\selectfont\char252}}1
{э}{{\selectfont\char253}}1
{ю}{{\selectfont\char254}}1
{я}{{\selectfont\char255}}1
{А}{{\selectfont\char192}}1
{Б}{{\selectfont\char193}}1
{В}{{\selectfont\char194}}1
{Г}{{\selectfont\char195}}1
{Д}{{\selectfont\char196}}1
{Е}{{\selectfont\char197}}1
{Ё}{{\"E}}1
{Ж}{{\selectfont\char198}}1
{З}{{\selectfont\char199}}1
{И}{{\selectfont\char200}}1
{Й}{{\selectfont\char201}}1
{К}{{\selectfont\char202}}1
{Л}{{\selectfont\char203}}1
{М}{{\selectfont\char204}}1
{Н}{{\selectfont\char205}}1
{О}{{\selectfont\char206}}1
{П}{{\selectfont\char207}}1
{Р}{{\selectfont\char208}}1
{С}{{\selectfont\char209}}1
{Т}{{\selectfont\char210}}1
{У}{{\selectfont\char211}}1
{Ф}{{\selectfont\char212}}1
{Х}{{\selectfont\char213}}1
{Ц}{{\selectfont\char214}}1
{Ч}{{\selectfont\char215}}1
{Ш}{{\selectfont\char216}}1
{Щ}{{\selectfont\char217}}1
{Ъ}{{\selectfont\char218}}1
{Ы}{{\selectfont\char219}}1
{Ь}{{\selectfont\char220}}1
{Э}{{\selectfont\char221}}1
{Ю}{{\selectfont\char222}}1
{Я}{{\selectfont\char223}}1
}


\usepackage{pdfpages}
\usepackage{longtable, multirow, rotating, color, colortbl}
\usepackage{amsmath}
\usepackage{tikz}
\usepackage{pgfplots}

\begin{document}

\frontmatter % выключает нумерацию ВСЕГО; здесь начинаются ненумерованные главы: реферат, введение, глоссарий, сокращения и прочее

% Команды \breakingbeforechapters и \nonbreakingbeforechapters
% управляют разрывом страницы перед главами.
% По-умолчанию страница разрывается.

% \nobreakingbeforechapters
% \breakingbeforechapters

\tableofcontents

\Defines % Необходимые определения. Вряд ли понадобться
\begin{description}
\item[Распределённый] Слово, которое нельзя употреблять. Но надо протестировать длинные строки в глоссарии.
\end{description}

%%% Local Variables:
%%% mode: latex
%%% TeX-master: "rpz"
%%% End:

\Abbreviations %% Список обозначений и сокращений в тексте
\begin{description}
\item[API] Интерфейс программирования приложений (англ. application programming interface)
\end{description}

%%% Local Variables:
%%% mode: latex
%%% TeX-master: "rpz"
%%% End:


\Introduction

В настоящее время всемирная паутина предоставляет огромное количество информации для пользователя на любой вкус. Миллиарды байт буквально лежать на кончиках пальцев миллионов пользователей. И в наше время не обязательно быть привязанным к компьютеру, чтобы получить доступ к этой информации. Достаточно иметь телефон, планшет или даже фотоаппарат с поддержкой беспроводной технологии доступа в интернет и ты получаешь возможность получить любую информацию за пару прикосновений прохладного стекла твоего девайса. Но протокол HTTP, благодаря которому мы имеем сегодня поистине безграничные возможности, разработан, как говорит название, для передачи гипертекста, то есть, данные приходят клиенту в виде размеченного документа. Это очень удобно при чтении документа с экрана компьютера или другого устройства отображения информации человеку, но совершенно не приемлемо для использования в других автоматизированных системах, когда клинтом является другое приложения. В таком случае язык HTML, который в основном и используется для распространения информации во всемирной паутине, является декоратором необходимых данных. 

При разработке приложений разработчики сталкиваются в первую очередь с проблемой получения и хранения информации. Первая проблема решается путем использования специализированных интерфейсов для получения данных и заботливые разработчики начинают все больше внимания уделять разработке API для того, чтобы их сервисы были удобнее как для простых пользователей, так и для программистов, желающих использовать накопившуюся у них ценную информацию в собственных целях.

Но так бывает далеко не во всех случаях, и в такой ситуации разработчику приходится разгребать кучи HTML кода для фильтрации необходимых данных и отсеивания лишней разметки для того, чтобы превратить тонны неструктурированной информации с просторов паутины в четко структурированные единицы данных. В таком случае очень удобна автоматизация процесса накопления, фильтрации и дальнейшего сохранения необходимой информации на стороне сервиса. 

Целью дипломной работы является создание программного комплекса для предоставления справочной информации для разработчиков о спортивных мероприятиях. Для достижения данной цели необходимо выполнить следующее: 

\begin{itemize}
\item разработать подсистему сбора информации, распространяемой в сети Internet в неструктурированном виде, в формате HTML страниц, работающую в реальном времени;
\item спроектировать базу данных для хранения собранной информации;
\item спроектировать REST API для предоставления доступа к собранной информации в удобном для использования виде;
\item проверить работоспособность системы путем создания приложения, используемого в качестве клиента API.
\end{itemize}

Подсистема сбора информации должна работать в качестве "демона" на сервере и представлять из себя очередь заданий для выполнения задач сбора. Собранная информация должна сохраняться в спроектированной базе данных через API для большей надежности и безопасности системы. Так же API должно предоставлять доступ на чтение для сторонних разработчиков, заинтересованных в хранимой информации. Клиентское приложение должно представлять собой web-сайт, предоставляющий пользователю доступ к статистической и исторической информации о спортивных мероприятиях. 

В данной работе упор делается на футбольную статистику, но разрабатываемая система должна быть легко расширяем для сбора и сохранения информации по другим видам спорта. Таким образом, каждая подсистема должна быть гибка и открыта для расширения.


\mainmatter % это включает нумерацию глав и секций в документе ниже

\chapter{Аналитический раздел}
\label{cha:analysis}
%
% % В начале раздела  можно напомнить его цель
%
В данном разделе производится анализ процессов сбора информации, сохранения её на стороне сервера и распространение её средствами API.
Производится анализ подсистем, входящих в реализуемы программный комплекс, формируются требования к создаваемой системе, выделяются функции её подсистем и описывается взаимодействие между ними.

\section{Общее описание системы}
В данной работе разрабатывается система полного цикла сбора, хранения и распространения информации, для функционирования которой необходимо спроектировать все подсистемы программного комплекса и способы их взаимодействия.

Для достижения поставленной цели необходимо спроектировать и разработать следующие компоненты системы:

\begin{enumerate}
\item подсистему сбора информации, находящихся на просторах интернета в виде страниц HTML, работающую в фоновом режиме;
\item базу данных для хранения собранной информации и быстрого доступа к ней;
\item простой и удобный в использовании REST API для предоставления доступа к собранной информации сторонним приложениям;
\item клиентское приложение, используемого в качестве клиента API для демонстрации возможностей разработанной.
\end{enumerate}

Так же для удобства и простоты использования необходимо разработать доступную документацию разработанного API, для того, чтобы любой желающий разработчик имел возможность легко и быстро воспользоваться услугами сервиса. 

Поскольку конечным пользователем разрабатываемой системы будет некое стороннее приложение, рассмотрим работу клиента системы с точки зрения разработчика-пользователя API. Работа с системой выглядит следующим образом:

\begin{enumerate}
\item разработчик создает приложение, имеющее возможность доступа в интернет;
\item он регистрирует свое приложение в системе для получения доступа к API;
\item используя известный протокол прикладного уровня передачи данных в среде web, приложение обращается по определенному адресу к API;
\item при обмене сообщениями протокола запрашиваемая система(сервер) отсылает ответ клиенту;
\item клиентское приложение использует полученную информацию по своему усмотрению.
\end{enumerate}

\section{Обзор существующих решений}
В настоящее время существует некоторое количество подобных решений, предоставляющих API(REST, SOAP, XML-RPC, и т.д.) для доступа к спортивной информации сторонним приложениям.
Но они ограничены в том или ином виде. Наиболее полные и интересные решения являются платными или условно бесплатными, предоставляющими пробный, ограниченный функционал бесплатно. Бесплатные API решения предоставляют неполную информацию или плохо документированный сервис. В большинстве API отсутствует информации о российских турнирах, что является одним из решающих факторов в пользу создания собственного API.

Наиболее интересными решениями в сфере предоставления спортивной информации являются следующие сервисы:

\begin{itemize}
\item \url{http://developer.espn.com/} - API был разработан для того, чтобы быть простым в использовании. Поддерживает XML, JSON/JSONP форматы ответа сервера. Имеет хорошую документацию. К сожалению нацелен на американского разработчика, так как в основном предоставляет доступ к информации, связанной с наиболее распространенными видам спорта в америке. Имеет условно бесплатный доступ;
\item \url{http://www.footytube.com/openfooty/} - Интересный сервис предоставляющие REST API возвращающий ответ в формате XML. Предоставляет информацию только о футболе. Хорошо документирован. Имеет ограничение в 5000 запросов в день;
\item \url{http://www.championat.com/} - Популярный российский спортивный портал, содержащий большое количество статистической и исторической информации, новостей из мира спорта, информации о текущих и будущих событиях. Имеет широкий круг рассматриваемых видов спорта, охватывающий широкий круг пользователей. Имеет элементы социальной сети. К сожалению не имеет собственного API;
\item ...
\end{itemize}

Ни один из этих сервисов не предоставляет информации о спортивных событиях в режиме "live".

Таким образом, обзор существующих решений показал, что все сервисы, предоставляющие подобный функционал являются довольно узкоспециализированными, ограниченными и не универсальными, а сервисы, предоставляющие необходимые данные в полном или практически полном объеме не предоставляют прямого доступа к ним посредствам API. 

\section{Подсистема сбора данных}
\subsection{Общие представления о системе}
 % Обратите внимание, что включается не ../dia/..., а inc/dia/...
% В Makefile есть соответствующее правило для inc/dia/*.pdf, которое
% берет исходные файлы из ../dia в этом случае.

Подсистема сбора данных должна решать проблемы "переваривания" "сырых" данных из сети и формирования из них базы данных. Таким образом, собранная на просторах интернета информация должна принять удобный для хранения, использования и распространения, структурированный вид. Данных процесс в общем случае представлен на рисунке 1.1.
\begin{figure}
  \centering
  \includegraphics[width=\textwidth]{inc/dia/analysis1-1}
  \caption{Рисунок}
  \label{fig:fig01}
\end{figure}


\subsection{Существующие сервисы сбора информации}

\begin{itemize}
\item \url{http://www.mozenda.com/} - сервис сбора информации в интернете. Является SaaS-приложением для решения схожих проблем поиска информации в сети. Является довольно дорогим сервисом. Предоставляет on-line конструктор web-пауков для сбора информации, что является несомненным плюсом сервиса
\item \url{http://www.fetch.com/} - название сайта говорит само за себя. Является похожим сервисом сбора информации с просторов паутины, но не предоставляет полной информации о своем сервисе и условиях использования на сайте. Для полного ознакомления предлагает связаться со службой поддержки.
\end{itemize}

Приведенный список показывает, что сбор информации в интернете является популярной и сложной задачей. Для универсального решения данной задачи требуется спроектировать сложный программно-аппаратный комплекс, требующий огромных вычислительных мощностей, что выходит за рамки данной работы и может воплотиться в виде развития рассматриваемой темы. 

\subsection{Процесс сбора информации}

Подсистема сбора информации должна работать в режиме 24/7 на некотором сервере и должна предоставлять возможность сбора информации сразу с нескольких предполагаемых источников.
Система сбора должна быть полностью автоматизирована и получать от администратора только правила фильтрации данных. Так как данная система работает в сети интернет, она должна быть готова к "отказам" серверных участников обмена информацией(источников данных), то есть при невозможности выполнения задачи получения страницы, система должна отложить данную задачу либо немедленно повторить процесс. Поскольку системе задаются некоторые правила сбора, она должна фиксировать промежуточные результаты работы в некотором хранилище. Такой подход позволит провести декомпозицию процесса сбора на элементарные единицы работы, что позволит более гибко реагировать на отказы системы, нештатные ситуации, связанные с работой сети и другие возможные сложности при выполнения сбора единицы полезной информации.

Данные требования хорошо решаются путем создания очереди заданий, представляющих собой систему массового обслуживания. Данная система состоит из очереди задач и обслуживающего аппарата. Поскольку процесс сбора единицы информации проходит декомпозицию на подзадачи, данные подзадачи должны ставиться на обслуживание в очередь отдельно. Таким образом мы получаем очередь с разными типами заявок. В случае простого накопления данных очередность выполнения заданий не имеет значения, так как итоговой целью работы очереди является выполнение всех задач. В нашем случае некоторые задачи могут иметь более высокий приоритет, так как должны быть доступны сразу после момента поступления(с некоторым допущением).
Данное требование естественным образом вводит свойство приоритета задачи. Таки образом, мы получаем очередь с приоритетами и, поскольку, данная работа не требует немедленного выполнения(из-за того же допущения, связанного со спецификой распространения информации в сети), то допускается ожидание завершения выполнения заявки, находящейся на обслуживании.
Данные требования приводят нас к решению использования очереди заданий с относительными приоритетами. 

На более низком уровне процесс сбора единицы информации выглядит следующим образом:
\begin{enumerate}
\item Система переходит по некоторой базовой ссылке, ведущей на определенный ресурс в интернете;
\item Полученный ответ система пытается разобрать по некоторым, заранее заданным правилам;
\item В случае нахождения очередной ссылки, в очередь ставится новая задача;
\item Процесс продолжается до тех пор, пока выполнение разбора не приведет к итоговой цели единицы информации;
\item В случае сбоя в процессе выполнения задачи, она ставится в очередь для повторного выполнения;
\end{enumerate}

Данный процесс представлен на рисунке 1.2.
Подсистема сбора данных должна решать проблемы "переваривания" "сырых" данных из сети и формирования из них базы данных. Таким образом, собранная на просторах интернета информация должна принять удобный для хранения, использования и распространения, структурированный вид. Данных процесс в общем случае представлен на рисунке 1.1.
\begin{figure}
  \centering
  \includegraphics[width=\textwidth]{inc/dia/analysis1-2}
  \caption{Рисунок}
  \label{fig:fig02}
\end{figure}

\chapter{Конструкторский раздел}
\label{cha:design}

В данном разделе проектируется новая всячина.

\section{Архитектура всячины}

\paragraph{Проверка} параграфа. Вроде работает.
\paragraph{Вторая проверка} параграфа. Опять работает.

Вот.

\begin{itemize}
\item Это список с <<палочками>>.
\item Хотя он и не по ГОСТ, кажется.
\end{itemize}

\begin{enumerate}
\item Поэтому для списка, начинающегося с заглавной буквы, лучше список с цифрами.
\end{enumerate}

Формула \ref{F:F1} совершено бессмысленна.

%Кстати, при каких-то условиях <<удавалось>> получить двойный скобки вокруг номеров формул. Вопрос исследуется.

\begin{equation}
a= cb
\label{F:F1}
\end{equation}


Окружение \texttt{cases} опять работает (см. \ref{F:F2}), спасибо И. Короткову за исправления..


\begin{equation}
a= \begin{cases}
 3x + 5y + z, \mbox{если хорошо} \\
 7x - 2y + 4z, \mbox{если плохо}\\
 -6x + 3y + 2z, \mbox{если совсем плохо}
\end{cases}
\label{F:F2}
\end{equation}

\section{Подсистема всякой ерунды}

Культурная вставка dot-файлов через утилиту dot2tex (рис.~\ref{fig:fig02}).

% \begin{figure}
%   \centering
% % [width=0.5\textwidth] --- регулировка ширины картинки
%   \includegraphics{inc/dot/cow2}
%   \caption{Рисунок}
%   \label{fig:fig02}
% \end{figure}


\subsection{Блок-схема всякой ерунды}

\subsubsection*{Кстати о заголовках}

У нас есть и \Code{subsubsection}. Только лучше её не нумеровать.

%%% Local Variables:
%%% mode: latex
%%% TeX-master: "rpz"
%%% End:

\chapter{Технологический раздел}
\label{cha:impl}

В данном разделе описано изготовление и требование всячины. Благодаря пакет \Code{underscore} эскейпить подчёркивание  не нужно (\Code{some_function}).

Для вставки кода есть пакет \Code{listings}. К сожалению, пакет \Code{listings} всё ещё
работает криво при появлении в листинге русских букв и кодировке исходников utf-8.
В данном примере он (увы) на лету конвертируется в koi-8 в ходе сборки pdf.

Есть альтернатива \Code{listingsutf8}, однако она работает лишь с
\texttt{\textbackslash lstinputlisting}, но не с окружением \Code{lstlisting}

Вот так можно вставлять псевдокод (питоноподобный язык определен в шаблоне):

\begin{lstlisting}[style=pseudocode,caption={Алгоритм оценки дипломных работ}]
def EvaluateDiplomas():
    for each student in Masters:
        student.Mark := 5
    for each student in Engineers:
        if Good(student):
            student.Mark := 5
        else:
            student.Mark := 4
\end{lstlisting}

Еще в шаблоне определен псевдоязык для BNF:

\begin{lstlisting}[style=grammar,basicstyle=\small,caption={Грамматика}]
  ifstmt -> "if" "(" expression ")" stmt |
            "if" "(" expression ")" stmt1 "else" stmt2
  number -> digit digit*
\end{lstlisting}

В листинге~\ref{lst:sample01} работают русские буквы. Сильная магия. Однако, работает
только во включаемых файлах, прямо в \TeX{} нельзя.

% Обратите внимание, что включается не ../src/..., а inc/src/...
% В Makefile есть соответствующее правило для inc/src/*,
% которое копирует исходные файлы из ../src и конвертирует из UTF-8 в KOI8-R.
% Кстати, поэтому использовать можно только русские буквы и ASCII,
% весь остальной UTF-8 вроде CJK и египетских иероглифов -- нельзя.

% \lstinputlisting[language=C,caption=Пример (\Code{test.c}),label=lst:sample01]{inc/src/test.c}

% Для вставки реального кода лучше использовать \texttt{\textbackslash lstinputlisting} (который понимает
% UTF8) и стили \Code{realcode} либо \Code{simplecode} (в зависимости от размера куска).




Можно также использовать окружение \Code{verbatim}, если \Code{listings} чем-то не
устраивает. Только следует помнить, что табы в нём <<съедаются>>. Существует так же команда \Code{verbatiminput} для вставки файла.

\begin{verbatim}
a_b = a + b; // русский комментарий
if (a_b > 0)
    a_b = 0;
\end{verbatim}

%%% Local Variables:
%%% mode: latex
%%% TeX-master: "rpz"
%%% End:

\chapter{Экспериментальный раздел}
\label{cha:research}

В данном разделе проводятся вычислительные эксперименты.
А на рис.~\ref{fig:spire01} показана схема мыслительного процесса автора...

% \begin{figure}
%   \centering
%   \includegraphics[width=\textwidth]{inc/svg/pic01}
%   \caption{Как страшно жить}
%   \label{fig:spire01}
% \end{figure}


%%% Local Variables:
%%% mode: latex
%%% TeX-master: "rpz"
%%% End:

\chapter{Организационно-экономический раздел}

Организационно-экономическая часть процесса разработки программного продукта предусматривает выполнение следующих работ:
\begin{itemize}
\item формирование состава выполняемых работ и группировка их по стадиям разработки;
\item расчет трудоемкости выполнения работ;
\item установление профессионального состава и расчет количества исполнителей;
\item определение продолжительности выполнения отдельных этапов разработки;
\item построение календарного графика выполнения разработки;
\item контроль выполнения календарного графика.
\end{itemize}

\section{Формирование состава выполняемых работ и группировка их по стадиям разработки}

Разработку программного продукта можно разделить на следующие стадии:

\begin{itemize}
\item техническое задание;
\item расчет трудоемкости выполнения работ;
\item эскизный проект;
\item технический проект;
\item рабочий проект;
\item внедрение.
\end{itemize}

Допускается объединение технического и рабочего проекта в технорабочий проект.

Планирование длительности этапов и содержания проекта осуществляется в соответствии с ЕСПД ГОСТ 34.603--92 и распределяет работы по этапам. как показано в таблице \ref{tab:jobsAndStages}.

\newpage
\footnotesize
\begin{longtable}{|l|c|p{0.65\textwidth}|}
    \caption{Распределение работ проекта по этапам}
    \label{tab:jobsAndStages}
        \\ \hline
        \multicolumn{1}{|l|}{\centering Основные стадии}
      & \multicolumn{1}{|c|}{\centering \No}
      & \multicolumn{1}{|p{0.5\textwidth}|}{\centering Содержание  работы} \\
        \hline
            \endfirsthead
        
        \subcaption{\normalsize Продолжение таблицы~\ref{tab:jobsAndStages}}
        \\ \hline \endhead
        \subcaption{\normalsize Продолжение на следующей странице}
        \endfoot
        \hline
        \endlastfoot
        
        \multirow{2}{*}{\centering 1. Техническое задание} & 1 & Постановка задачи \\
        \cline{2-3}
        & 2 & Выбор средств проектирования и разработки \\
        \hline
        \multirow{4}{*}{\centering 2. Эскизный проект} & 3 & Разработка структуры системы \\
        \cline{2-3}
        & 4 & Разработка алгоритмов описания моделей и моделирования \\
        \cline{2-3}
        & 5 & Разработка вспомогательных алгоритмов \\
        \cline{2-3}
        & 6 & Разработка пользовательского интерфейся \\
        \hline
        \multirow{7}{*}{\centering 2. Технорабочий проект} & \centering 7 & Реализация алгоритмов описания моделей и моделирования\\
        \cline{2-3}
        & 8 & Реализация вспомогательных алгоритмов \\
        \cline{2-3}
        & 9 & Реализация пользовательского интерфейса \\
        \cline{2-3}
        & 10 & Отладка программного продукта \\
        \cline{2-3}
        & 11 & Исправление ошибок и недочетов \\
        \cline{2-3}
        & 12 & Разработка документации к системе \\
        \cline{2-3}
        & 13 & Итоговое тестирование системы \\
        \hline
        4. Внедрение & 14 & Установка и настройка программного продукта \\
        \hline
\end{longtable}
\normalsize

\section{Расчет трудоемкость выполнения работ}

Трудоемкость разработки программной продукции заывисит от ряда факторов, основными из которых являются следующие:

\begin{itemize}
\item степень новизны разрабатываемого программного продукта;
\item сложность алгоритма его функционирования;
\item объем используемой информации, вид ее представления и способ обработки;
\item уровень используемого алгоритмического языка программирования.
\end{itemize}

Разрабатываемый программный продукт можно отнести:

\begin{itemize}
\item по степени новизны~--- к категории В. Разрботка программной продукции имеющей аналоги. 
\item по степени сложности алгоритма функционирования~--- к 1-ой группе (программная продукиця реализующая моделирующие алгоритмы). 
\end{itemize}

Трудоемкость разработки программного продукта $\tau_{\text{ПП}}$ может быть определена как сумма величин трудоемкости выполнения отдельных стадий разработки ПП из выражения~\ref{F:tayPP}.

\begin{equation}
\tau_{\text{ПП}} = \tau_{\text{ТЗ}} + \tau_{\text{ЭП}} + \tau_{\text{ТП}} + \tau_{\text{РП}} + \tau_{\text{В}}
\label{F:tayPP}
\end{equation}, где

$\tau_{\text{ТЗ}}$~--- трудоемкость разработки технического задания; $\tau_{\text{ЭП}}$~--- трудоемкость разработки эскизного проекта; $\tau_{\text{ТП}}$~--- трудоемкость разработки технического проекта; $\tau_{\text{РП}}$~--- трудоемкость разработки рабочего проекта; $\tau_{\text{В}}$~--- трудоемкость внедрения.

Трудоемкость разработки технического задания рассчитывается по формуле~\ref{F:tayTZ}

\begin{equation}
\tau_{\text{ТЗ}} = T_{\text{ЗРЗ}} + T_{\text{ЗРП}}
\label{F:tayTZ}
\end{equation}, где

$T_{\text{ЗРЗ}}$~--- затраты времени разработчика постановки задач на разработку ТЗ, чел.-дни; $T_{\text{ЗРП}}$~--- затраты времени разработчика ПО на раззработку ТЗ, чел.-дни.

Значения величин $T_{\text{ЗРЗ}}$ и $T_{\text{ЗРЗ}}$ рассчитываются по формулам \ref{F:TZRZ} и \ref{F:TZRP}.

\begin{equation}
T_{\text{ЗРЗ}} = t_{\text{З}} \cdot K_{\text{ЗРЗ}}
\label{F:TZRZ}
\end{equation}

\begin{equation}
T_{\text{ЗРП}} = t_{\text{З}} \cdot K_{\text{ЗРП}}
\label{F:TZRP}
\end{equation}, где

$t_{\text{З}}$~--- норма времени на разработку ТЗ на ПП в зависимости от его функционального назначенияя и стпени новизны, чел.-дни; $K_{\text{ЗРЗ}}$~--- коэффициент, учитывающий удельный вес трудоемкости работ, выполняемых разработчикм ТЗ; $K_{\text{ЗРЗ}}$~--- коэффициент, учитывающий удельный вес трудоемкости работ, выполняемых разраьотчиком ПО на стадии ТЗ.

$t_{\text{З}} = 24$~чел.-дн. (управление НИР)

$K_{\text{ЗРЗ}} = 0.65$ (совместная разработка)

$K_{\text{ЗРП}} = 0.35$ (совместная разработка)

$\tau_{\text{ТЗ}} = 24 \cdot 0.65 + 24 \cdot 0.35 = 24$~чел.-дн.

Аналогично рассчитывается трудоемкость эскизного проекта $\tau_{\text{ЭП}}$:

$\tau_{\text{ЭП}} = 70 \cdot 0.5 + 70 \cdot 0.5 = 70$~чел.-дн.

Трудоемкость разработки технического проекта $\tau_{\text{ТП}}$ зависит от функционального назначения ПП, количества разновидностей форм входной и выходной информации и определяется как сумма времени, затраченного разрабьотчикм постановки задач и разработчиком программного обеспечения по формуле~\ref{F:tauTP}.

\begin{equation}
\tau_{\text{ТП}} = (t_{\text{ТРЗ}} + t_{\text{ТРП}}) \cdot K_{\text{В}} \cdot K_{\text{Р}}
\label{F:tauTP}
\end{equation}, где

$t_{\text{ТРЗ}}$ и $t_{\text{ТРП}}$~--- норма времени, затрачивваемого на разработку ТП разрабьотчиком постановки задач и разработчиком программного обуспечения соответственно, чел.-дни; $K_{\text{В}}$~--- коэффициент учета вида используемой информации, $K_{\text{Р}}$~--- коэффициент учета режима обработки информации.

Значение коэффициента $K_{\text{В}}$ определяется из выражения:

\begin{equation}
K_{\text{В}} = \frac{K_{\text{П}} \cdot n_{\text{П}} + K_{\text{НС}} \cdot n_{\text{НС} + K_{\text{Б}} \cdot n_{\text{Б}}}}{n_{\text{П}} + n_{\text{НС}} + n_{\text{Б}}}
\end{equation}

где  $K_{\text{П}}$, $K_{\text{НС}}$, $K_{\text{Б}}$~--- значения коэффициентов учета вида используемой информации для переменной, нормативно-справочной информации и баз данных соответственно; $n_{\text{П}}$, $n_{\text{НС}}$, $n_{\text{Б}}$~--- количество наборов данных переменной, нормативно-справочной информации и баз данных соответственно.

$K_{\text{П}} = 1$, $K_{\text{НС}}$ - 0.72, $K_{\text{Б} = 2.08}$

$K_{\text{В}} = \frac{1 \cdot 3 + 0.72 \cdot 1 + 2.08 \cdot 0}{3 + 1 + 0} = 0.505$

$K_{\text{Р}} = 1.26$

$\tau_{\text{ТП}} = (33 + 10) \cdot 0.505 \cdot 1.26 = 28$, чел.-дн.

Трудоемкость разработки рабочего проекта $\tau_{\text{РП}}$ зависит от функционального назначения ПП, количества разновидностей форм входной информации, сложности алгоритма функционирования, сложности контроля информации, степени использования готовых программных модулей, уровня алгоритмического языка программирования и определяется по формуле:

\begin{equation}
\tau_{\text{РП}} = K_{\text{К}} \cdot K_{\text{Р}} \cdot K_{\text{Я}} \cdot K_{\text{З}} \cdot K_{\text{ИА}} \cdot (t_{\text{РРЗ}} + t_{\text{РРП}})
\label{F:tauRP}
\end{equation}

где $K_{\text{К}}$~--- коэффициент учета сложности контроля информации; $K_{\text{Я}}$~--- коэффициент учета уровня используемого алгоритмического языка программирования; $K_{\text{З}}$~--- коэффициент учета степени использования готовых программных модулей; $K_{\text{ИА}}$~--- коэффициент учета вида используемой информации и сложности алгоритма ПП.

$K_{\text{К}} = 1, K_{\text{Р}} = 1.44, K_{\text{Я}} = 1, K_{\text{З}} = 0.7, t_{\text{РРЗ}} = 9$~чел.-дн., $t_{\text{РРП}} = 54$~чел.-дн., $K_{\text{П}} = 1.2, K_{\text{НС}} = 0.65, K_{\text{Б}} = 0.54$

$K_{\text{ИА}} = \frac{1.2 \cdot 3 + 0.65 \cdot 1 + 0.54 \cdot 0}{3 + 1 + 0} = 1.06$

$\tau_{\text{РП}} = 1 \cdot 1.44 \cdot 1 \cdot 0.7 \cdot 1.06 \cdot (9 + 54) = 67$

Так как при разработке ПП стадии <<Технический проект>> и <<Рабочий проект>> объеденины в стадию <<Техно-рабочий проект>>, то трудоемоксть ее выполнения $\tau_{\text{ТРП}}$ определяется по формуле:

\begin{equation}
\tau_{\text{ТРП}} = 0.85 \cdot (\tau_{\text{ТП}} + \tau_{\text{РП}})
\label{F:tauTRP}
\end{equation}
$\tau_{\text{ТРП}} = 0.85 \cdot (28 + 67) = 91$


Трудоемкость выполнения стадии внедрения $\tau_{\text{В}}$ может быть расчитана по формуле:

\begin{equation}
\tau_{\text{В} = (t_{\text{ВРЗ}} + t_{\text{ВРП}}) \cdot K_{\text{К}} \cdot K_{\text{Р}} \cdot K_{\text{З}}}
\label{F:tauV}
\end{equation}

где $t_{\text{ВРЗ}}$, $t_{\text{ВРП}}$~--- норма времени, затрачиваемого разработчиком постановки задач и разработчиком ПО соответственно на выполнение процедур внедрения ПП, чел.-дни.

$\tau_{\text{В}} = (10 + 11) \cdot 1 \cdot 1.26 \cdot 0.7 = 19$ чел.-дн.

Подставив полученные данные в формулу~\ref{F:tayPP} получим:

$\tau_{\text{ПП}} = 24 + 70 + 91 + 19 = 204$ чел.-дн.

\begin{table}[ht]\footnotesize
    \caption{Распределение трудоемкости по стадиям разработки проекта}
    \begin{tabular}{|c|c|c|p{0.70\textwidth}|c|}
    \hline
    \begin{sideways}Этап\end{sideways} &
    \begin{sideways} \parbox{30mm}{Трудоемкость \\этапа, чел.-дн. }\end{sideways} &
    \No & \multicolumn{1}{p{0.7\textwidth}|}{\centering Содержание работы}&
    \begin{sideways} \parbox{30mm}{Трудоемкость, чел.-дн. }\end{sideways} \\
    \hline
    \multirow{2}{*}{\centering 1} & \multirow{2}{*}{\centering 24} & 1 & Постановка задачи, разработка ТЗ & 20 \\
    \cline{3-5}
    & & 2 & Выбор средств проектирования и разработки & 4\\
    \hline
    \multirow{4}{*}{\centering 2} & \multirow{4}{*}{\centering 70} & 3 & Разработка структуры системы & 15 \\
    \cline{3-5}
    & & 4 & Разработка алгоритмов описания модели и моделирования & 30\\
    \cline{3-5}
    & & 5 & Разработка вспомогательных алгоритмов & 15\\
    \cline{3-5}
    & & 6 & Разработка пользовательского интерфейса & 10\\
    \hline
    \multirow{7}{*}{\centering 3} & \multirow{7}{*}{\centering 91} & 7 & Реализация алгоритмов описания модели и моделирования & 23 \\
    \cline{3-5}
    & & 8 & Реализация вспомогательных алгоритмов & 16\\
    \cline{3-5}
    & & 9 & Реализация пользовательского интерфейса & 10\\
    \cline{3-5}
    & & 10 & Отладка программного продукта & 12\\
    \cline{3-5}
    & & 11 & Исправление ошибок и недочетов & 15\\
    \cline{3-5}
    & & 12 & Разработка документации к системе & 8\\
    \cline{3-5}
    & & 13 & Итоговое тестирование системы & 7\\
    \hline
    4 & 19 & 14 & установка и настройка ПП & 19 \\
    \hline
    & & &{\raggedleft{Итого:}} & 204 \\
    \hline
    \end{tabular}
\end{table}

\normalsize

\section{Расчет количества исполнителей}

Средняя численность исполнителей при реализации проекта разработки и внедрения ПО определяется соотношением:

\begin{equation}
N = \frac{Q_{\text{Р}}}{F}
\label{F:N}
\end{equation}

где $Q_{\text{Р}}$~--- затраты труда на выполнение проекта (разработка и внедрение ПО); $F$~--- фонд рабочего времени.

Величина фонда рабочего времени определяется соотношением:

\begin{equation}
F = T \cdot F_M
\label{F:F}
\end{equation}

где $T$~--- фвремя выполнения проекта в месяцах, равное 4 месяцам; $F_M$~--- фонд времени в текущем месяце, который рассчитывается из учета числа дней в году, числа выходных и праздничных дней:

\begin{equation}
F_M = \frac{t_{\text{р}} \cdot (D_{\text{К}} - D_{\text{В}} - D_{\text{П}})}{12}
\label{F:FM}
\end{equation}

где $t_{\text{р}}$~--- продолжительность рабочего дня; $D_{\text{К}}$~--- общее число дней в году; $D_{\text{В}}$~--- число выходных дней в году; $D_{\text{П}}$~--- число праздничных дней в году.

$F_M = \frac{8 \cdot (365 - 103 - 10)}{12} = 168$

$F = 4 \cdot 168 = 672$

$N = \frac{204 \cdot 8}{672} = 3$~--- число исполнителей проекта.

\section{Календарный план-график}

\begin{table}[ht]\footnotesize
\caption{Планирование разработки}
\begin{tabular}{|p{0.2\textwidth}|l|p{0.25\textwidth}|p{0.17\textwidth}|l|}
\hline
Стадия разработки & Трудоемкость & Должность исполнителя& Распределение трудоемкости & Численность \\
\hline
\multirow{2}{0.2\textwidth}{Техническое задание} & \multirow{2}{*}{24} & Ведущий программист & 18(75\%) & 1\\
& & Программист 1 & 6(25\%) & 1 \\
\hline
\multirow{3}{0.2\textwidth}{Эскизный проект} & \multirow{3}{*}{70} & Ведущий программист & 26(37\%) & 1\\
& & Программист 1 & 27(39\%) & 1 \\
& & Программист 2 & 17(24\%) & 1 \\
\hline
\multirow{3}{0.2\textwidth}{Технорабочий проект} & \multirow{3}{*}{91} & Ведущий программист & 36(40\%) & 1\\
& & Программист 1 & 32(35\%) & 1 \\
& & Программист 2 & 23(25\%) & 1 \\
\hline
\multirow{2}{0.2\textwidth}{Внедрение} & \multirow{3}{*}{19} & Ведущий программист & 8(42\%) & 1\\
& & Программист 2 & 11(58\%) & 1 \\
\hline
\end{tabular}
\end{table}
\normalsize

\begin{table}[ht]\footnotesize
\caption{Календарный ленточный график работ}
\begin{tabular}{|l|l|llll|llll|llll|llll|}
\hline
Стадия разработки & Должность исполнителя & \multicolumn{16}{c|}{Трудоемкость} \\ 
\hline
\multirow{2}{0.2\textwidth}{Техническое задание} & Ведущий программист & \multicolumn{4}{l|}{\cellcolor[gray]{0.6}18} & & & & & & & & & & & &\\
& Программист 1 & \multicolumn{2}{l|}{\cellcolor[gray]{0.8}6} & {} & {} & & & & & & & & & & & & \\
\hline
\multirow{3}{0.2\textwidth}{Эскизный проект} & Ведущий программист & & & & & \multicolumn{4}{l|}{\cellcolor[gray]{0.8}26} & & & & & & & & \\
& Программист 1 & & & & & \multicolumn{4}{l|}{\cellcolor[gray]{0.6}27} & & & & & & & &\\
& Программист 2 & & & & &  \cellcolor[gray]{0.8}7 & {} & {} & \cellcolor[gray]{0.8}7 & & & & & & & & \\
\hline
\multirow{3}{0.2\textwidth}{Технорабочий проект} & Ведущий программист & & & & & & & & &\multicolumn{4}{l|}{\cellcolor[gray]{0.6}36} & & & & \\
& Программист 1 & & & & & & & & &\multicolumn{4}{l|}{\cellcolor[gray]{0.8}32} & & & & \\
& Программист 2 & & & & & & & & & {} & \multicolumn{3}{l|}{\cellcolor[gray]{0.8}23} & {} & & & \\
\hline
\multirow{2}{0.2\textwidth}{Внедрение} & Ведущий программист & & & & & & & & & & & & &\multicolumn{4}{l|}{\cellcolor[gray]{0.8}9} \\
& Программист 1 & & & & & & & & & & & & & \multicolumn{4}{l|}{\cellcolor[gray]{0.6}10}\\
\hline
\end{tabular}
\label{tab:timeline}
\end{table}

\normalsize 

Из таблицы~\ref{tab:timeline} видно, что благодаря параллельной работе ведущего программиста и программистов можно добиться сокращения сроков разработки с  204 дней до $18 + 27  + 36 + 11 = 92$ дней, т.е. в 2.2 раза.

\section{Расчет затрат на разработку ПП}

Затраты на выполнение проекта сосстоят из затрат на заработную плату исполнителям, затрат на закупку или аренду оборудования, затрат на организацию рабочих мест и затрат на накладные расходы. В таблице~\ref{tab:zarp} приведены затраты на заработную плату и страховые взносы во внебюджетные фонды. Суммарно эти отчисления для органиаций, осуществляющих деятельность в области информационных технологий, составляют 14\%: в Пенсионный фонд (ПФ)~--- 8\%, в Фонд социального стразования (ФСС), Федеральный и территориальный фонды обязательного медицинского страхования (ФФОМС и ТФОМС)~--- по 2\%. Для ведущего программиста предполагается ставка 2500 рублей за полный рабочий день, для первого программиста~--- 1500 рублей, для второго~--- 1000 рублей. 

\begin{table}[ht!]\footnotesize
\caption{Затраты на зарплату и отчисления во внебюджетные фонды}
\begin{tabular}{|l|l|l|l|l|l|l|}
\hline
& \multicolumn{6}{c|}{Февраль}\\
\hline
Исполниель & Рабочих дней & Зарплата & ПФ & ФСС & ФФОМС & ТФОМС\\
\hline
Ведущий программист & 20 & 50000 & 4000 & 1000 & 1000 & 1000\\
\hline
Программист 1 & 20 & 30000 & 2400 & 600 & 600 & 600\\
\hline
Программист 2 & 19 & 19000 & 1520 & 380 & 380 & 380\\
\hline
Итого: & \multicolumn{6}{r|}{112860}\\
\hline
& \multicolumn{6}{c|}{Март}\\
\hline
Исполниель & Рабочих дней & Зарплата & ПФ & ФСС & ФФОМС & ТФОМС\\
\hline
Ведущий программист & 22 & 55000 & 4400 & 1100 & 1100 & 1100\\
\hline
Программист 1 & 22 & 33000 & 2640 & 660 & 660 & 660\\
\hline
Программист 2 & 6 & 6000 & 480 & 120 & 120 & 120\\
\hline
Итого: & \multicolumn{6}{r|}{107160}\\
\hline
& \multicolumn{6}{c|}{Апрель}\\
\hline
Исполниель & Рабочих дней & Зарплата & ПФ & ФСС & ФФОМС & ТФОМС\\
\hline
Ведущий программист & 21 & 52500 & 4200 & 1050 & 1050 & 1050\\
\hline
Программист 1 & 21 & 31500 & 2520 & 630 & 630 & 630\\
\hline
Программист 2 & 10 & 10000 & 800 & 200 & 200 & 200\\
\hline
Итого: & \multicolumn{6}{r|}{107160}\\
\hline
& \multicolumn{6}{c|}{Май}\\
\hline
Исполниель & Рабочих дней & Зарплата & ПФ & ФСС & ФФОМС & ТФОМС\\
\hline
Ведущий программист & 20 & 50000 & 4000 & 1000 & 1000 & 1000\\
\hline
Программист 1 & 20 & 30000 & 2400 & 600 & 600 & 600\\
\hline
Программист 2 & 19 & 19000 & 1520 & 380 & 380 & 380\\
\hline
Итого: & \multicolumn{6}{r|}{112860}\\
\hline
& \multicolumn{6}{c|}{Июнь}\\
\hline
Исполниель & Рабочих дней & Зарплата & ПФ & ФСС & ФФОМС & ТФОМС\\
\hline
Ведущий программист & 16 & 40000 & 3200 & 800 & 800 & 800\\
\hline
Программист 1 & 15 & 22500 & 1800 & 450 & 450 & 450\\
\hline
Программист 2 & 3 & 3000 & 240 & 60 & 60 & 60\\
\hline
Итого: & \multicolumn{6}{r|}{74670}\\
\hline
Общая сумма: & \multicolumn{6}{r|}{446310}\\
\hline
\end{tabular}
\label{tab:zarp}
\end{table}
\normalsize

Расходы на материалы, необходимые для разработки ПП указаны в таблице~\ref{tab:materials}.

\begin{table}[ht!]\footnotesize
\caption{Материальные затраты}
\begin{tabular}{|c|p{0.25\textwidth}|l|c|c|c|}
\hline
\No & Наименование материала & Единица измерения & Кол-во & Цена за единицу, руб. & Сумма, руб.\\
\hline
1 & Бумага А4 & Пачка 500 л. & 1 & 150 & 150 \\
\hline
2 & Картридж для лазерного принтера Brother HL-2030R & Шт. & 2 & 1400 & 2800 \\
\hline
\multicolumn{5}{|r|}{Итого:} & 2950 \\
\hline
\end{tabular}
\label{tab:materials}
\end{table}

При разработке понадобится 3 компьютера (по одному на человека), один принтер, компьютерная мебель (3 комплекта), аксесуары для компьютеров (3 компьютера). Стоимость одной ПЭВМ составляет 35000 рублей. Месячная норма амортизации $K = 4\%$ (Срок службы~--- 25 месяцев). Срок службы составляет 3 года, месячная амортизация $K = 2.78\%$. Срок службы компьютерной мебели: столы~--- 5 лет, стулья~--- 2 года, месячные нормы амортизации~--- $K = 1.67\%$ и $K = 4.17\%$ соответственно. компьютерные аксесуары имеют срок службы 2 года, норма амортизации $K = 4.17\%$.

Затраты на амортизацию приведены в таблице~\ref{tab:amort}

\begin{table}[ht]\footnotesize
\caption{Амортизационные отчисления}
\begin{tabular}{|l|c|c|c|c|c|}
\hline
\parbox{20mm}{Наименование \\ оборудования} & \parbox{20mm}{Балансовая\\цена, руб.} & Кол-во, шт. & К, \% & {Время использования, мес.} & {Сумма отчислений, руб.} \\
\hline
Компьютер & 35000 & 3 & 4 & 4 & 16800 \\
\hline
Принтер & 2600 & 1 & 2.78 & 4 & 290 \\
\hline
Стол & 3000 & 3 & 1.67 & 4 & 600 \\
\hline
Стул & 1000 & 3 & 4.17 & 4 & 500 \\
\hline
Клавиатура & 500 & 3 & 4.17 & 4 & 350 \\
\hline
Мышь & 500 & 3 & 4.17 & 4 & 350 \\
\hline
\multicolumn{5}{|r|}{Итого:} & 18790 \\
\hline
\end{tabular}
\label{tab:amort}
\end{table}
\normalsize

Таке необходимо учесть аренднуюю полату за помещение, которая составляет 14000 рублей за месяц. За 4 месяца на аренду помещения уйдет 56000 рублей.

Общие затраты на разработку ПП составят $446310 + 2950 + 18790 + 56000 = 524050$ рублей.

\section{Расчет экономической эффективности}

Основными показателями экономической эффективности является чистый дисконтированный доход (ЧДД) и сок окупаемости вложенных средств.

Чистый дисконтированный доход определяется по формуле:

\begin{equation}
\text{ЧДД} = \sum_{t=0}^{T}{R_t - \text{З}_t \cdot \frac{1}{(1 + E)^t}}
\label{F:ChDD}
\end{equation}

где $T$~--- горизонт рачета по месяцам; $t$~--- период расчета; $R_t$~--- результат достигнутый на шаге (стоимость)$t$; $\text{З}_t$~--- затраты; $E$~--- приемлимая для инвестора норма прибыли на вложенный капитал.

Коэффициент $E$ установим равным ставке рефинансирования ЦБ РФ~--- 8\% годовых (0.67\% в месяц). В результате анализа рынка программной продукции, аналогичной разрабатваемой, планируется продажа 2 единц в 1-й месяц и 3 единицы в остальные. Планируемая цена ПП составляет 60000 рублей. Предполагаемые накладные расходы, связанные с реализацией составят 2000 рублей в месяц.

В таблице~\ref{tab:ChDD} приведен расчет ЧДД по месяцам работы над проектом, а на Рисунке~\ref{fig:ChDD} изображен график ЧДД (начерчен до момента, когда ЧДД принимает положительные значения).

\begin{table}[ht]\footnotesize
\caption{Расчет ЧДД}
\begin{tabular}{|c|c|c|c|c|c|}
\hline
Месяц & \parbox{20mm}{Текущие затраты, руб.}& \parbox{30mm}{Затраты с начала \\года, руб.}&\parbox{20mm} {Текущий \\доход, руб.}& \parbox{30mm}{Доход с начала\\года, руб.}& ЧДД, руб.\\
\hline
1 & 66451 & 66451 & 0 & 0 &-66451\\
\hline
2 & 126201 & 192652 & 0 & 0 &-192652\\
\hline 
3 & 126201 & 318853 & 0 & 0 &-318853\\
\hline
4 & 131901 & 450754 & 0 & 0 & -450754\\
\hline
5 & 73296 & 524050 & 60000 & 60000 &-464050\\
\hline
6 & 2000 & 526050 & 180000 & 240000 & -286050\\
\hline
7 & 2000 & 528050 & 180000 & 420000 & -108050\\
\hline
7 & 2000 & 530050 & 180000 & 600000 & 69950\\
\hline
\end{tabular}
\label{tab:ChDD}
\end{table}
\normalsize

 Срок окупаемости проекта~--- 8 месяцев.
 
\begin{figure}[ht]
 \centering
 \begin{tikzpicture}
    \begin{axis}[
     width = 11cm,
     xlabel = {Время, мес.},
     ylabel = {ЧДД, руб.},
     xtick align = center,
     yminorgrids, ymajorgrids,
     xmajorgrids,
     minor y tick num = 4,
     legend style={at={(0.74,0.74)}, anchor=southwest}
    ],
    \addplot[black!80!black, mark=o, smooth] plot coordinates {
        (1,-66451)
        (2,-192652)
        (3,-318853)
        (4,-450754)
        (5,-464050)
        (6,-286050)
        (7,-108050)
        (8,69950)
    };
    \end{axis}
 \end{tikzpicture}
 \caption{График изменения ЧДД}
 \label{fig:ChDD}
\end{figure}


\section{Выводы}

По результатам расчета трудоемкости выполнения работ, затрат на выполнение работ и прибыли от реализации программного продукта был найден чистый дисконтированный доход по месяцам разработки и реализации программного продукта. На основе полученных данных можно сделать вывод, что проект эффективен и срок его окупаемости составляет 8 месяцев с начала разработки.
\chapter{Охрана труда и экология}

\section{Оценка условий труда на рабочем месте пользователя ПЭВМ}

Разработка ПО требует длительного взаимодействия с вычислительными системами. Работа с ПЭВМ связана с рядом вредных и опасных факторов, таких как статическое электричество, рентгеновское излучение, электромагнитные поля, блики отраженный свет, ультрафиолетовое излучение. При длительном воздействии на организм эти факторы негативно влияют на здоровье человека.

\subsection{Параметры микроклимата}

Параметры микроклимата могут меняться в широких пределах, в то время как необходимым условием жизнедеятельности человека является поддержание постоянства температуры тела благодаря терморегуляции, т.е. способности организма регулировать отдачу тепла в окружающую среду. Принцип нормирования микроклимата - создание оптимальных условий для теплообмена
тела человека с окружающей  средой. Вычислительная техника является источником существенных тепловыделений, что может привести к повышению температуры и снижению относительной влажности в помещении. В помещениях, где установлены компьютеры, должны соблюдаться определенные параметры микроклимата. Нормы, установленные СанПиН 2.2.2/2.4.1340-03 для категории работ 1а приведены в таблице~\ref{tab:microclimate}. Эти нормы устанавливаются в зависимости от времени года, характера трудового процесса и характера производственного помещения.

\begin{table}[ht]
\caption{Параметры микроклимата}
\begin{tabular}{|l|c|c|c|c|c|c|}
\hline
\multirow{2}{*}{Период год} & \multicolumn{2}{l|}{Температура, $^\circ \mbox{C}$} & \multicolumn{2}{l|}{Влажность, \%} & \multicolumn{2}{l|}{Скорость воздуха, м/с} \\
\cline{2-7}
&Оптим.&Допуст.&Оптим.&Допуст.&Оптим.&Допуст.\\
\hline
Холодный &22--24&21--25&40--60&75&0.1&0.1\\
\hline
Теплый &23--25&22--28&40--60&55 при 28$^\circ \mbox{C}$&0.1&0.1\\
\hline 
\end{tabular}
\label{tab:microclimate}
\end{table}

Вредным фактором при работе с ЭВМ является также запыленность помещения. Этот фактор усугубляется влиянием на частицы пыли электростатических полей персональных компьютеров.

Для устранения несоответствия параметров указанным нормам проектом предусмотренно использование системы кондиционирования как наиболее эффективного и автоматически функционирующего средства.

Нормы установленные содержания в воздухе положительных и отрицательной ионов, установленные СанПиН 2.2.4.1294--03, приведены в таблице~\ref{tab:ions}.

\begin{table}[ht]
\caption{Уровни ионизации воздуха при работе на ПЭВМ}
\begin{tabular}{|l|c|c|}
\hline
\multirow{2}{*}{Уровни} & \multicolumn{2}{l|}{Число ионов в кубометре воздуха}\\
\cline{2-3}
&$n^+$&$n^-$\\
\hline
Минимально необходимое & 400 & 600 \\
\hline
Оптимальное & 1500--3000 & 3000--5000 \\
\hline
Максимально допустимое & 50000 & 50000 \\
\hline
\end{tabular}
\label{tab:ions}
\end{table}

Для обеспечения требуемых уровней предусмотренно использование системы ионизации Сапфир-4А.

Объем помещений, в которых размещены работники вычислительных центров, не должен быть меньше $19.5 \frac{\text{м}^3}{\text{человека}}$ с учетом максимального числа одновременно работающих в смену. Нормы подачи свежего воздуха в помещения, где расположены компьютеры, приведены в таблице~\ref{tab:air}.

\begin{table}[ht]
\caption{Уровни ионизации воздуха при работе на ПЭВМ}
	\begin{tabular}{|l|l|}
	\hline
	Уровни & Число ионов в кубометре воздуха \\ 
	\hline
	Объем до $20 {\text{м}^3}$ на человека & Не менее 30 \\
	\hline
	$20.40 {\text{м}^3}$ на человека & Не менее 20 \\
	\hline
	Более $40 {\text{м}^3}$ на человека & Естественная вентиляция \\
	\hline
	\end{tabular}
\label{tab:air}
\end{table}

Для обеспечения комфортных условий используются как организационные методы (рациональная организация проведения работ в зависимости от времени года и суток, чередование труда и отдыха), так и технические средства (вентиляция, кондиционирование воздуха, отопительная система).

\subsection{Шум и вибрации}

Уровень шума на рабочем месте программиста не должен превышать 50 дБА, а уровень вибрации не должен превышать норм установленных СанПиН 2.2.2.542--96 (см. таблицу~\ref{tab:vibro}).

\begin{table}[ht]
\caption{Допустимые нормы вибрации на раочих местах с ПЭВМ}
\begin{tabular}{|c|c|c|}
\hline
\parbox{0.4\textwidth}{ Среднегеометрические частоты\\октавных полос, Гц}& \multicolumn{2}{l|}{Допустимые значения по виброскорости}\\
\cline{2-3}
&м/c &дБ\\
\hline
2  & $4.5\times10$ & 79 \\
\hline
4  & $2.2\times10$ & 73 \\
\hline
8  & $1.1\times10$ & 67 \\
\hline
16  & $1.1\times10$ & 67 \\
\hline
31.5 & $1.1\times10$ & 67 \\
\hline
63  & $1.1\times10$ & 67 \\
\hline
\parbox{0.4\textwidth}{ Корректированные значения\\и их уровни в дБ}& $2.0\times10$ & 72\\
\hline
\end{tabular}
\label{tab:vibro}
\end{table}

При разработке ПО внутренними источниками шума являются вентиляторы, а также принтеры и другие перефферийные устройства ЭВМ. Внешние источники шума~--- прежде всего, шум с улицы и из соседних помещений. Постоянные внешние источники шума, превышающего нормы, отсутствуют.

Для устранения превышения нормы проектом предусмотрено применение звукопоглощающих материалов для облицовки стен и потолка помещения, в котором осуществляется работа с вычислительной техникой.

\subsection{Освещение}

Наиболее важным условием эффективной работы программистов и пользователей является соблюдение оптимальных параметров системы освещения в рабочих помещениях.

Естественное освещение осуществляется через светопроемы, ориентированные в основном на север и северо-восток (для исключения попадания прямых солнечных лучей на экраны компьютеров) и обеспечивает коэффициент естественной освещенности (КЕО) не ниже 1.5\%.

В качестве искусственного освещения проектом предусмотрено использование системы общего освещения. в соответствии с СанПин 2.2.2/2.4.1340--03 освещенность на поверхности рабочего стола должна находиться в пределах 300--500 лк. Разрешается использование светильников местного освещения для работы в документами (при этом светильники не должны создавать блики на поверхности экрана).

Правильное расположение рабочих мест относительно источников освещения, отсутствие зеркальных поверхностей и использование матовых материалов ограничивает прямую (от источников освещения) и отраженную (от рабочих поверхностей) блескость. При  этом яркость светящихся поверхностей не превышает $200 \frac{\text{кд}}{\text{м}^2}$, яркость бликов на экране ПЭВМ не превышает $40 \frac{\text{кд}}{\text{м}^2}$, и яркость потолка не превышает $200 \frac{\text{кд}}{\text{м}^2}$.

В соответствии с СанПинН 2.2.2/2.4.1340--03 проектом предусмотрено использование люминесцентных ламп типа ЛБ в качестве источников света при искусственном освещении. В светильниках допускается применение ламп накаливания. Применение газоразрядных ламп в светильниках общего и местного освещения обеспечивает коэффициент пульсации не более 5\%.

Таким образом, проектом обеспечиваются оптимальные условия освещения рабочего помещения.

\subsection{Рентгеновское излучение}

В соответствии с СанПиН 2.2.2/2.4.1340-03 проектом предусмотрено использование ПЭВМ, конструкция которого обеспечивает мощность экспозиционной дозы рентгеновского излучения в любой точке на расстоянии 0.5 м. от экрана и корпуса не более 0.1 мбэр/час (100 мкР/час). Результаты сравнения норм излучения приведены в таблице~\ref{tab:rentgen}.

\begin{table}[ht]
\caption{Сравнение норм рентгеновского излучения в различных стандартах}
\begin{tabular}{|l|c|}
\hline
& Допустимое значение мкР/час, не более \\
\hline
СанПиН 2.2.2/2.4.1340-03 & 100 \\
\hline
ТСО-99 & 500 \\
\hline
MPR II & 500\\
\hline
\end{tabular}
\label{tab:rentgen}
\end{table}

Как видно из таблицы, стандарты MPR II и ТСО--99 предъявляют менее жесткие требования к рентгеновскому излучению, чем СанПиН. Но при соблюдении оптимального расстояния между пользователем и монитором дозы рентгеновского излучения не опасны для большинства людей.

\subsection{Неионизирующие электромагнитные излучения}

Допустимые значения параметров неионизирующих излучений в соответствии с СанПин 2.2.2/2.4.1340-03 приведены в таблицах~\ref{tab:U} и~\ref{tab:ro}.

\begin{table}[ht]
\caption{Предельно допустимые значения напряженности электрического поля}
\begin{tabular}{|c|c|}
\hline
Диапазон частот& Допустимые значения \\
\hline
5 Гц -- 2 кГц & 25 В/м \\
\hline
2 -- 400 кГц& 2.5 В/м \\
\hline
\end{tabular}
\label{tab:U}
\end{table}

\begin{table}[ht]
\caption{Предельно допустимые значения плотности магнитного потока}
\begin{tabular}{|c|c|}
\hline
Диапазон частот& Допустимые значения \\
\hline
5 Гц -- 2 кГц & 250 нТл \\
\hline
2 -- 400 кГц& 5 нТл \\
\hline
\end{tabular}
\label{tab:ro}
\end{table}

Величина поверхностного электрического потенциала не должна превышать 500 В.

Мониторы, используемые в настоящее время, удовлетворяют более жестким нормам MPR II, а значит и СанПиН.

\subsection{Визуальные параметры}

Неправильный выбор визуальных эргономических параметров приводит к ухудшению здоровья пользователей, быстрой утомляемости, раздражительности. В связи с этим, проектом предусмотрено, что конструкция вычислительной системы и ее эргономические параметры обеспечивают комфортное и надежное считывание информации. Требования к визуальным параметрам, их внешнему виду, дизайну, возможности настройки представлены в СанПиН 2.2.2/2.4.1340--03. Визуальные эргономические параметры монитора и пределы из изменений приведены в таблице~\ref{ergonom}.

\begin{table}[ht]
\caption{Визуальные эргономические параметры ВДТ и пределы из изменений}
\begin{tabular}{|l|c|c|}
\hline
\multirow{2}{*}{Наименование параметров} & \multicolumn{2}{c|}{Пределы значений параметров}\\
\cline{2-3}
&не менее&не более\\
\hline
Яркость экрана (фона), $\frac{\text{кд}}{\text{м}^2}$ (измеренная в темноте) &35&120\\
\hline
Внешняя освещенность экрана, лк &100&250\\
\hline
Угловой размер экрана, угл.мин. &16&60\\
\hline
\end{tabular}
\label{tab:ergonom}
\end{table}

Для выполнения этих требований проектом предусмотренно использование современных мониторов, имеющих достаточно широкий набор регулируемых параметров.  В частности, для удобного считывания информации реализована возможность настройки положения монитора по горизонтали и вертикали. Мониторы оснащены специальными устройствами и средствами настройки ширины, высоты, яркости, контраста и разрешения изображения. кроме того, в современных мониторах зерно изображения имеет размер в пределах 0.27 мм, что обеспечивает высокую четкость и непрерывность изображения. Наконец, на поверхность дисплея нанесено матовое покрытие, чтобы избавиться от солнечных бликов.

\section{Расчет искусственного освещения}

При расчете освещенности от светильников общего равномерного освещения наиболее часто применяют метод расчета по световому потоку. При расчете освещения по этому методу необходимое количество светильников для освещения рабочего места рассчитывается по формуле:

\begin{equation}
\label{f:lightsCount}
N = \frac{E_{min}\cdot S\cdot K}{F_\text{Л} \cdot \text{З} \cdot z \cdot h}
\end{equation}

где $E_{min}$~--- нормируемая минимальная освещенность; $S$~--- площадь помещения, $\text{м}^2$; $F_\text{Л}$~--- световой поток лампы, лк; $K$~--- коэффициент запаса; $z$~--- коэффициент неравномерности освещения (для люминесцентных ламп~---1.1); $h$~--- коэффициент использования светового потока в долях единицы.

$E_{min}$ определяется на основании нормативного документа СНиП23--05--95. В соответствии с произведенным выбором в предыдущем разделе, для работы программиста $E_{min}=300$ лк (общее освещение).

Работы, производятся в помещении, требуют различения цветных объектов при невысоких требованиях к цветоразличению, поэтому в качестве источника освещения была выбрана лампа люминесцентная холодно-белая (ЛХБ), 1940 лк, 30 Вт. В помещениях общественных и жилых зданий с нормальными условиями среды: К=1.4.

Для люминесцентных ламп коэффициент неравномерности освещения Z=1.1.

Коэффициент использования h зависит от типа светильника, от коэффициентов отражения потолка $\rho_\text{П}$, стен $\rho_\text{С}$, расчетной поверхности $\rho_\text{Р}$ и индекса помещения.

Высота подвеса над рабочей поверхностью Нр=3 м. Размеры помещения А=3.5 м, В=3 м. Определим индекс помещения по формуле:

\begin{equation}
\phi = \frac{A \cdot B}{H_P \cdot (A + B)} = \frac{3.5 \cdot 3}{s \cdot (3.5 + 3)} = 0.54
\end{equation}

Для светлого фона примем:$\rho_\text{П} = 70$ $\rho_\text{С} = 50$ $\rho_\text{Р} = 10$. h = 59\%.

Освещение проектируется при помощи светильников ОДОР с минимальной освещенностью $E_{min}=300$ лк, P=40 Вт. Число ламп в ОДОР равно 2. Необходимое число светильников для данной комнаты:

\begin{equation}
N = \frac{300 \cdot 9 \cdot 1.4}{1940 \cdot 0.59 \cdot 1.1 \cdot 2} = 2 \text{шт}
\end{equation}

Общее количество ламп $n = 2\times2=4$ шт. Длина светильника ОДОР=1.26 м. Поскольку длина помещения 3 м, то светильники помещаются в два ряда. 
Суммарная мощность светильников: $30\cdot4=160$ Вт. Сумарный световой поток: $1940\cdot4=7760$ лм.

\section{Режим труда}
При работе с персональным компьютером очень важную роль играет соблюдение правильного режима труда и отдыха. В противном случае у программиста отмечаются значительное напряжение зрительного аппарата с появлением жалоб на неудовлетворенность работой головные боли, раздражительность, нарушение сна, усталость и болезненные ощущения в глазах, в пояснице, в области шеи и руках.

При несоответствии фактических условий труда требованиям санитарных правил и норм время регламентированных перерывов следует увеличить на 30\%. В соответствии со СанПиН 2.2.2 546-96 все виды трудовой деятельности, связанные с использованием компьютера, разделяются на три группы: 
\begin{enumerate}
	\item группа А: работа по считыванию информации с экрана ВДТ или ПЭВМ с предварительным запросом; 
	\item группа Б: работа по вводу информации; 
	\item группа В: творческая работа в режиме диалога с ЭВМ. 
\end{enumerate}

Режим труда и отдыха должен зависеть от характера работы: при вводе данных, редактировании программ, чтении информации с экрана непрерывная продолжительность работы с монитором не должна превышать 4 часов. При 8 часовом рабочем дне, через каждый час работы необходимо проводить перерыв 5--10 минут, а каждые два часа перерыв в 15 мин.

Эффективность перерывов повышается при сочетании с производственной гимнастикой или организации специального помещения для отдыха персонала с удобной мягкой мебелью, аквариумом, зеленой зоной и т.п.


\backmatter %% Здесь заканчивается нумерованная часть документа и начинаются ссылки и
            %% заключение

\Conclusion % заключение к отчёту

В результате проделанной работы стало ясно, что ничего не ясно...

%%% Local Variables: 
%%% mode: latex
%%% TeX-master: "rpz"
%%% End: 


% % Список литературы при помощи BibTeX
% Юзать так:
%
% pdflatex rpz
% bibtex rpz
% pdflatex rpz

\bibliographystyle{gost780u}
\bibliography{rpz}

%%% Local Variables: 
%%% mode: latex
%%% TeX-master: "rpz"
%%% End: 


\appendix   % Тут идут приложения

\chapter{Картинки}
\label{cha:appendix1}

\begin{figure}
\centering
\caption{Картинка в приложении. Страшная и ужасная.}
\end{figure}

%%% Local Variables: 
%%% mode: latex
%%% TeX-master: "rpz"
%%% End: 

\chapter{Еще картинки}
\label{cha:appendix2}

\begin{figure}
\centering
\caption{Еще одна картинка, ничем не лучше предыдущей. Но надо же как-то заполнить место.}
\end{figure}

%%% Local Variables: 
%%% mode: latex
%%% TeX-master: "rpz"
%%% End: 


\end{document}

%%% Local Variables:
%%% mode: latex
%%% TeX-master: t
%%% End:
