\chapter{Аналитический раздел}
\label{cha:analysis}
%
% % В начале раздела  можно напомнить его цель
%
В данном разделе производится анализ процессов сбора информации, сохранения её на стороне сервера и распространение её средствами API.
Производится анализ подсистем, входящих в реализуемы программный комплекс, формируются требования к создаваемой системе, выделяются функции её подсистем и описывается взаимодействие между ними.

\section{Общее описание системы}
В данной работе разрабатывается система полного цикла сбора, хранения и распространения информации, для функционирования которой необходимо спроектировать все подсистемы программного комплекса и способы их взаимодействия.

Для достижения поставленной цели необходимо спроектировать и разработать следующие компоненты системы:

\begin{enumerate}
\item подсистему сбора информации, находящихся на просторах интернета в виде страниц HTML, работающую в фоновом режиме;
\item базу данных для хранения собранной информации и быстрого доступа к ней;
\item простой и удобный в использовании REST API для предоставления доступа к собранной информации сторонним приложениям;
\item клиентское приложение, используемого в качестве клиента API для демонстрации возможностей разработанной.
\end{enumerate}

\begin{figure}
  \centering
  \includegraphics[width=\textwidth]{inc/dia/analysis1-1}
  \caption{Схема работы системы}
  \label{fig:fig01}
\end{figure}

Так же для удобства и простоты использования необходимо разработать доступную документацию разработанного API, для того, чтобы любой желающий разработчик имел возможность легко и быстро воспользоваться услугами сервиса. 

Поскольку конечным пользователем разрабатываемой системы будет некое стороннее приложение, рассмотрим работу клиента системы с точки зрения разработчика-пользователя API. Работа с системой выглядит следующим образом:

\begin{enumerate}
\item разработчик создает приложение, имеющее возможность доступа в интернет;
\item он регистрирует свое приложение в системе для получения доступа к API;
\item используя известный протокол прикладного уровня передачи данных в среде web, приложение обращается по определенному адресу к API;
\item при обмене сообщениями протокола запрашиваемая система(сервер) отсылает ответ клиенту;
\item клиентское приложение использует полученную информацию по своему усмотрению.
\end{enumerate}

\section{Обзор существующих решений}
В настоящее время существует некоторое количество подобных решений, предоставляющих API(REST, SOAP, XML-RPC, и т.д.) для доступа к спортивной информации сторонним приложениям.
Но они ограничены в том или ином виде. Наиболее полные и интересные решения являются платными или условно бесплатными, предоставляющими пробный, ограниченный функционал бесплатно. Бесплатные API решения предоставляют неполную информацию или плохо документированный сервис. В большинстве API отсутствует информации о российских турнирах, что является одним из решающих факторов в пользу создания собственного API.

Наиболее интересными решениями в сфере предоставления спортивной информации являются следующие сервисы:

\begin{itemize}
\item \url{http://developer.espn.com/} - API был разработан для того, чтобы быть простым в использовании. Поддерживает XML, JSON/JSONP форматы ответа сервера. Имеет хорошую документацию. К сожалению нацелен на американского разработчика, так как в основном предоставляет доступ к информации, связанной с наиболее распространенными видам спорта в америке. Имеет условно бесплатный доступ;
\item \url{http://www.footytube.com/openfooty/} - Интересный сервис предоставляющие REST API возвращающий ответ в формате XML. Предоставляет информацию только о футболе. Хорошо документирован. Имеет ограничение в 5000 запросов в день;
\item \url{http://www.championat.com/} - Популярный российский спортивный портал, содержащий большое количество статистической и исторической информации, новостей из мира спорта, информации о текущих и будущих событиях. Имеет широкий круг рассматриваемых видов спорта, охватывающий широкий круг пользователей. Имеет элементы социальной сети. К сожалению не имеет собственного API;
\item ...
\end{itemize}

Ни один из этих сервисов не предоставляет информации о спортивных событиях в режиме "live".

Таким образом, обзор существующих решений показал, что все сервисы, предоставляющие подобный функционал являются довольно узкоспециализированными, ограниченными и не универсальными, а сервисы, предоставляющие необходимые данные в полном или практически полном объеме не предоставляют прямого доступа к ним посредствам API. 

\section{Подсистема сбора данных}
\subsection{Общие представления о системе}

Подсистема сбора данных должна решать проблемы "переваривания" "сырых" данных из сети и формирования из них базы данных. Таким образом, собранная на просторах интернета информация должна принять удобный для хранения, использования и распространения, структурированный вид. Результатом работы такой системы должен быть массив данных, готовый к автоматизированной обработке. В идеальном случае данная подсистема должна быть изменяемой легко настраиваемой для различных целей, типов данных, и доменных областей, но в данной работе она рассматривается в разрезе сбора данных о спортивных мероприятиях. Данных процесс в общем случае представлен на рисунке 1.2.
\begin{figure}
  \centering
  \includegraphics[width=\textwidth]{inc/dia/analysis1-2}
  \caption{Подсистема сбора данных}
  \label{fig:fig02}
\end{figure}

\subsection{Существующие сервисы сбора информации}

\begin{itemize}
\item \url{http://www.mozenda.com/} - сервис сбора информации в интернете. Является SaaS-приложением для решения схожих проблем поиска информации в сети. Является довольно дорогим сервисом. Предоставляет on-line конструктор web-пауков для сбора информации, что является несомненным плюсом сервиса
\item \url{http://www.fetch.com/} - название сайта говорит само за себя. Является похожим сервисом сбора информации с просторов паутины, но не предоставляет полной информации о своем сервисе и условиях использования на сайте. Для полного ознакомления предлагает связаться со службой поддержки.
\end{itemize}

Приведенный список показывает, что сбор информации в интернете является популярной и сложной задачей. Для универсального решения данной задачи требуется спроектировать сложный программно-аппаратный комплекс, требующий огромных вычислительных мощностей, что выходит за рамки данной работы и может воплотиться в виде развития рассматриваемой темы. 

\subsection{Возможности системы}
Подсистема сбора данных должна обеспечивать непрерывное автоматическое накопление существующих, а так же постоянно появляющихся новых данных в сети интернет. Таким образом приложение, реализующее данную систему, должно постоянно работать на сервере. Данная подсистема должны выполнять сбор следующих данных:
\begin{enumerate}
\item информация о видах спорта
  \begin{enumerate}
    \item название
    \item основной сезон проведения соревнований
    \item проводимые соревнования
    \item участники соревнований
  \end{enumerate}
\item информация о соревновании
  \begin{enumerate}
    \item название соревнования
    \item годы проведения
    \item страна проведения
    \item участники соревнования
    \item результаты проведения
  \end{enumerate}
\item информация о матчах
  \begin{enumerate}
    \item команда-хозяин
    \item команда-гость
    \item результат матча
    \item статистика матча
  \end{enumerate}
\item информация об участниках
  \begin{enumerate}
    \item название команды
    \item состав команды
    \item результаты выступления
    \item статистика выступления
    \item составы команд
    \item личная статистика игроков
    \item сводная статистика выступлений команды
    \item страна, тренер и другая информация о команде
  \end{enumerate}
\end{enumerate}

Данная информация впоследствии может быть использована для составления сводной статистики, выявления закономерностей выступлений определенных спортсменов, поиска интересных фактов о той или иной команде или отдельном игроке, а так же для любой другой автоматической обработке данных.

Подсистема сбора данных должна иметь web-интерфейс для удобной настройки, администрирования и профилирования работы очереди.
Данных интерфейс должен предоставлять следующие возможности:

\begin{enumerate}
  \item управление видами спорта
  \item управление странами проведения соревнований
  \item управление проводимыми соревнованиями 
  \item управление задачами для выполнения
  \item отслеживание состояния очереди
  \item остановка/запуск очереди
  \item просмотр сводной статистики работы очереди
  \item уведомление о событиях системы
  \item просмотр логов
\end{enumerate}

Для выполняемых задач должна быть возможность отложенного выполнения, задания приоритета выполнения, возможности добавления и удаления задач. Поскольку очередь должна работать в автоматическом режиме, то очередь должна быть сконфигурирована только на начальном этапе. Затем администратор может только добавлять новые соревнования для разбора, добавляя в очередь адрес страницы для инициализации разбора.

\subsection{Процесс сбора информации}

Подсистема сбора информации должна работать в режиме 24/7 на некотором сервере и должна предоставлять возможность сбора информации сразу с нескольких предполагаемых источников.
Система сбора должна быть полностью автоматизирована и получать от администратора только правила фильтрации данных. Так как данная система работает в сети интернет, она должна быть готова к "отказам" серверных участников обмена информацией(источников данных), то есть при невозможности выполнения задачи получения страницы, система должна отложить данную задачу либо немедленно повторить процесс. Поскольку системе задаются некоторые правила сбора, она должна фиксировать промежуточные результаты работы в некотором хранилище. Такой подход позволит провести декомпозицию процесса сбора на элементарные единицы работы, что позволит более гибко реагировать на отказы системы, нештатные ситуации, связанные с работой сети и другие возможные сложности при выполнения сбора единицы полезной информации.

Данные требования хорошо решаются путем создания очереди заданий, представляющих собой систему массового обслуживания. Данная система состоит из очереди задач и обслуживающего аппарата. Поскольку процесс сбора единицы информации проходит декомпозицию на подзадачи, данные подзадачи должны ставиться на обслуживание в очередь отдельно. Таким образом мы получаем очередь с разными типами заявок. В случае простого накопления данных очередность выполнения заданий не имеет значения, так как итоговой целью работы очереди является выполнение всех задач. В нашем случае некоторые задачи могут иметь более высокий приоритет, так как должны быть доступны сразу после момента поступления(с некоторым допущением).
Данное требование естественным образом вводит свойство приоритета задачи. Таким образом, мы получаем очередь с приоритетами и, поскольку, данная работа не требует немедленного выполнения(из-за того же допущения, связанного со спецификой распространения информации в сети), то допускается ожидание завершения выполнения заявки, находящейся на обслуживании.
Данные требования приводят нас к решению использования очереди заданий с относительными приоритетами. 

На более низком уровне процесс сбора единицы информации выглядит следующим образом:
\begin{enumerate}
\item Система переходит по некоторой базовой ссылке, ведущей на определенный ресурс в интернете;
\item Полученный ответ система пытается разобрать по некоторым, заранее заданным правилам;
\item В случае нахождения очередной ссылки, в очередь ставится новая задача;
\item Процесс продолжается до тех пор, пока выполнение разбора не приведет к итоговой цели единицы информации;
\item В случае сбоя в процессе выполнения задачи, заявка на её выполнение ставится в очередь для повторного выполнения;
\item При окончании разбора данные сохраняются в некоторой системе хранения данных.
\end{enumerate}

Данный процесс представлен на рисунке 1.3.
\begin{figure}
  \centering
  \includegraphics[width=\textwidth]{inc/dia/analysis1-3}
  \caption{Схема работы подсистемы сбора данных}
  \label{fig:fig03}
\end{figure}

Выходом системы является набор данных, готовых для сохранения в некоторой системе хранения данных. Так как система выполняет непрерывное накопление данных, то приложение должно сохранять информацию о том, какие данные уже были разобраны и сохранены в системе. Для этих целей может служить временное хранилище, как и для статистической информации о работе очереди. Данная база данных может быть развернута на том же сервере, что и сама система и предоставлять удобный интерфейс для хранения информации.

Готовые данных должны отправлять на сервер баз данных, который может быть доступен для данного приложения с помощью некоторого API. Так как главной целью данной задачи является управление данными, то наиболее подходящим решением данной проблемы является использование REST API, а тот факт, что разработка данного вида программного интерфейса является одной из подцелей данной работы, то это решение является наиболее удобным. Таким образом приложение предоставления данных будет так же предоставлять интерфейс для наполнения его базы. После окончания сбора данные будут отправляться на сервер для последующей обработки и сохранения.

\subsection{Стадии обработки заявки очереди}
Рассмотрим процесс сбора статистики матча. В соответствии с требованиями к работе очереди и ненадежности работы сети очень важно убедиться в том, что все данные будут собраны. Для того чтобы получить некоторую единицу информации необходимо пройти несколько этапов. Рассмотрим этот процесс на примере сбора статистики о матче. 

Для получения необходимой информации в автоматическом режиме необходимо проделать следующие операции:
\begin{enumerate}
\item получить список сезонов, сыгранных в той или иной лиге по некоторому виду спорта;
\item получить список матчей для каждого сезона;
\item для каждого из матчей получить информацию о матче;
\item получить информацию о командах, участвующих в матче;
\item получить статистику игроков команд.
\end{enumerate}
Возьмем, например, 10 сезонов для Английской премьер лиги по футболу. В лиге участвует 20 команд. При форме проведения чемпионата <<каждый с каждым>> все команды встречаются между собой по два раза. Получаем 38 матчей для каждой команды за сезон или 380 матчей в лиге. При условии получения прямых ссылок на каждом шаге получается 391 запрос. Если хотя бы один из запросов потерпит неудачу, то в лучшем случае для опубликования собранной информации необходимо ждать полного сбора, в худшем - будет потеряна вся собранная информация.
Для того чтобы этого избежать, необходимо разбить проблему на подзадачи. При получении каждой новой порции информации, будет создаваться новая заявка для постановки в очередь. 

Декомпозиция этой задачи позволит безболезненно откладывать выполнение, ставить задачи для повторного выполнения в случае неудачи, а так же сократить время сбора итоговой единицы информации. 

Данный процесс представлен на рисунке 1.4.  
\begin{figure}
  \centering
  \includegraphics[width=\textwidth]{inc/dia/analysis1-4}
  \caption{Процесс сбора статистики матча}
  \label{fig:fig04}
\end{figure}


В процессе получения информации о матче можно так же получить и сопутствующую информацию, такую как информацию о командах участницах, составах команд, их игроках и т.д.

\begin{figure}
  \centering
  \includegraphics[width=\textwidth]{inc/dia/analysis1-5}
  \caption{Процесс сбора статистики команд и игроков}
  \label{fig:fig05}
\end{figure}

При таком разбиении на атомарные задачи, система может быть настроена на работу в любой доменной области при минимальных изменениях правил непосредственного разбора информации со страницы полученного сайта, представленного в формате HTML, либо какого-либо сервиса в любом ином формате, будь то JSON, XML или любой другой распространенный формат. Так как другой распространенной проблемой является сбор медиа данных, то есть скачивание изображений, видео либо иных файлов, эта задача отлично вписывается в данную схему. Таким образом такой подход к декомпозиции задачи сбора данных является гибким и масштабируемым решением.

\subsection{Требования к системе}
В соответствии с проведенным анализом системы и процессов в не сформулируем  требования к подсистеме сбора данных.

Т\subsection{Требования к реализации}

\begin{enumerate}
 \item Подсистема сбора должна выполняться в виде отдельного сервиса;
 \item Сбор данных должен выполняться в автоматическом режиме и требовать вмешательства оператора только в случае нештатной ситуации;
 \item Новые данные должны попадать на сервер распространения информации как можно быстрее;
 \item Она должна быть гибкой к изменениям в формате разметки данных на стороне источника данных;
 \item Должен быть разработан удобный интерфейс добавления новых источников данных;
 \item Параметры системы должны быть изменяемыми;
 \item Время ожидания ответа от источника данных должно задаваться в на стройках системы, а ошибки при работе источника не должны приводить к сбоям в работе очереди и потерям заявок;
 \item Подсистема сбора данных должна сохранять информацию об уже собранных данных для того, чтобы не производить повторных попыток разбора уже накопленных данных.
\end{enumerate}

\subsection{Требования к надежности системы}

\begin{enumerate}
 \item Подсистема сбора данных должна проектироваться с учетом отсутствия надежной связи со сторонними системами, а их сбои не должны приводить к некорректной работе подсистемы;
 \item Информация о заявках на обслуживание не должна теряться из очереди при боях в питании и связи;
 \item В случае возникновения сбоя система должна корректно продолжить обработку прервавшихся заявок;
 \item Подсистема должна корректно обрабатывать исключения взаимодействия с другими системами и в случае возникновения нештатных ситуаций, информировать об этом оператора; 
 \item Все заявки в системе должны доходить до финальной стадии своей обработки;
 \item Web-интерфейс администратора должен быть доступен при условии физической работоспособности серверов системы.
\end{enumerate}

\subsection{Входные данные подсистемы}

\begin{enumerate}
 \item Адреса систем источников данных, начиная с которых следует начать сбор той или иной порции информации;
 \item Адрес или адреса системы накопления информации для дальнейшей обработки и сохранения выполненной заявки;
 \item Правила разбора данных при получении их от источника информации;
 \item Правила разбора полученной информации для передачи ее подсистеме накопления данных;
 \item Начальные заявки на обработку;
 \item Типы задач и принимаемые ими исходные данные для выполнения;
 \item Приоритеты заявок в очереди.
\end{enumerate}

\subsection{Выходные параметры}
\begin{enumerate}
 \item Результат обработки заявки в виде некоторой структуры данных;
 \item В случае, если заявка не прошла полный цикл обработки, данные для постановки в очередь новой заявки на выполнение;
 \item Информация о состоянии заявки в системе.
\end{enumerate}

\section{Подсистема предоставления доступа к данных}

\subsection{Общие представления о системе}

Главной задачей разрабатываемой системы является предоставление доступа сторонним приложениям доступ к массивам данных, хранящихся на серверах системы. Как известно, "чтобы купить что-нибудь ненужное, нужно сначала купить что-нибудь ненужное", и эту задачу призвана решить первая рассматриваемая подсистема - подсистема сбора данных. Второй важнейшей задачей является сохранение данных, а так же и непосредственная передача клиенту. Эту проблему призвано решить вторая рассматриваемая подсистема.

Очевидно, что в качестве сердца данной подсистемы выступает некоторая база данных, которую необходимо наполнить данными, а затем и предоставить доступ к этим данным. Для решения это задачи должно быть разработано приложения, предоставляющий некоторый интерфейс для доступа к базе данных, так как прямой доступ к базе крайне нежелателен, а в некоторых случаях даже опасен. Такой интерфейс и будет предоставлять второе приложение.

\begin{figure}
  \centering
  \includegraphics[width=\textwidth]{inc/dia/analysis1-5}
  \caption{Схема работы подсистемы управления данными}
  \label{fig:fig05}
\end{figure}

Для предоставления доступа к данным сторонним клиентам, данное приложение должно предоставлять доступ к нему по сети. Таким образом оно представляет собой web-сервис с некоторым интерфейсом. Поскольку данное приложение в некотором роде является адаптером между клиентскими приложениями и построенной базой данных, то необходимо построить удобный интерфейс для данной доменной области. Как известно, главной задачей практически любой базы данных является выполнение CRUD-операций, т.е. операций над данными({\bf C}reate - создание, {\bf R}ead - чтение, {\bf U}pdate - изменение и {\bf D}elete - удаление). На данную модель работы с данными отлична ложится другая распространенная методология построения архитектуры распределенного приложения - REST.

REST (Representational state transfer) - это стиль архитектуры программного обеспечения для распределенных систем, таких как World Wide Web, который, как правило, используется для построения веб-служб. Приложения построенные по методологии REST представляют из себя клиент-серверные приложения, не сохраняющими состояние между запросом и ответом, а так же обеспечивают гибкость архитектуры и масштабируемость системы. В общем случае REST является очень простым интерфейсом управления информацией без использования каких-то дополнительных внутренних прослоек. Каждая единица информации однозначно определяется глобальным идентификатором, таким как URL. Каждая URL в свою очередь имеет строго заданный формат. Отсутствие дополнительных внутренних прослоек означает передачу данных в том же виде, что и сами данные. То есть, мы не заворачиваем данные в XML, как это делает SOAP и XML-RPC. Просто отдаем сами данные.

Каждая единица информации однозначно определяется URL - это значит, что URL по сути является первичным ключом для единицы данных. Например, лига, сохраненная в системе под номером 3 будет иметь вид /league/3, а первая команда этой лиги - /league/3/team/1. Отсюда и получается строго заданный формат. Причем совершенно не имеет значения, в каком формате находятся данные по адресу /league/3/team/1 - это может быть и HTML, JSON, XML или даже отсканированная таблица результатов из свежей газеты в формате jpeg.

Как происходит управление информацией сервиса - это целиком и полностью основывается на протоколе передачи данных. Наиболее распространенный протокол конечно же HTTP. Так вот, для HTTP действие над данными задается с помощью методов: GET (получить), PUT (добавить, заменить), POST (добавить, изменить, удалить), DELETE (удалить). Таким образом, действия CRUD могут выполняться как со всеми 4-мя методами, так и только с помощью GET и POST.

Вот как это будет выглядеть на примере:

\begin{itemize}
 \item GET /team/ - получить список всех команд
  \item GET /team/1/ - получить команду под первым номером
  \item PUT /team/ - добавить команду (данные в теле запроса)
  \item POST /team/1 - изменить команду (данные в теле запроса)
  \item DELETE /team/1 - удалить команду
\end{itemize}

POST может использоваться одновременно для всех действий изменения: 

\begin{itemize}
  \item POST /team/ - добавить команду (данные в теле запроса)
  \item POST /team/1 - изменить команду (данные в теле запроса)
  \item POST /team/1 - удалить команду (тело запроса пустое)
\end{itemize}


Это позволяет иногда обходить неприятные моменты, связанные с отсутствием поддержки браузерами методов PUT и DELETE.

Самое главное достоинство сервисов в том, что с ними работать может какая угодно система, будь то сайт, flash, программа и др. так как методы парсинга наиболее популярных форматов разметки данных и выполнения запросов HTTP присутствуют почти везде. 

Построение приложения по архитектуре REST позволяет серьезно упростить эту задачу. 

Таким образом, рассмотренная архитектура является идеальным выбором для решения нашей задачи и наш сервис буде предоставлять REST API для доступа к данным.

\subsection{Возможности системы}

\section{Клиентское приложение}
\subsection{Общие представления о системе}